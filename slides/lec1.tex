\documentclass[pdftex,landscape,final,handout,british]{beamer}
%\definecolor{verdeuni}{rgb}{0.7,0.73437,0.55856}
%\setbeamertemplate{headline}[verdeuni]
%\setbeamercovered{highly dinamic}
%\usepackage{eso-pic}

\definecolor{red}{rgb}{0.7,0.0,0.0}

\usepackage[utf8]{inputenc}
\usepackage[english]{babel}
\usepackage{color}
\usepackage{url}
\usepackage{hyperref}
\usepackage{fancyvrb}
\usepackage{tikz}
\usepackage{alltt}
\usepackage{etex, xy}
\usepackage{tipa}

\xyoption{all} %para los diagramas
\usepackage{cancel}
\usepackage{verbatim}
%\usepackage{slashbox}
\usetikzlibrary{decorations.pathreplacing,shapes.arrows}
\newcommand\BackgroundPicture[1]{%
  \setbeamertemplate{background}{%
   \parbox[c][\paperheight]{\paperwidth}{%
      \vfill \hfill
\includegraphics[width=1\paperwidth,height=0.9\paperheight]{#1}
        \hfill \vfill
     }}}

\usepackage{color}
\usepackage{listings}


\newcommand{\rrdc}{\mbox{\,\(\Rightarrow\hspace{-9pt}\Rightarrow\)\,}} % Right reduction
\newcommand{\lrdc}{\mbox{\,\(\Leftarrow\hspace{-9pt}\Leftarrow\)\,}}% Left reduction
\newcommand{\lrrdc}{\mbox{\,\(\Leftarrow\hspace{-9pt}\Leftarrow\hspace{-5pt}\Rightarrow\hspace{-9pt}\Rightarrow\)\,}} % Equivalence
\DeclareMathOperator{\id}{Id}
\newcommand{\sCoq}{{\sc Coq}}
\newcommand{\SSReflect}{{\sc SSReflect}}
\newcommand{\Zset}{\mathbb{Z}}
\newcommand{\Bset}{\mathbb{Z}_2}


\newtheorem{Thm}{Theorem}
\newtheorem{Prop}{Proposition}
\newtheorem{Lem}{Lemma}
\newtheorem{Cor}{Corollary}
\newtheorem{Defn}{Definition}
\newtheorem{Goal}{Goal}
\newtheorem{GGoal}{General Goal}
\newtheorem{Alg}{Algorithm}
\newcommand{\inp}{{\itshape Input: }}
\newcommand{\outp}{{\itshape Output: }}

\renewcommand{\thefootnote}{$\ast$}
\title[Haskell Lecture 1: An Introduction to Haskell]{AC21007: Haskell Lecture 1\\
An Introduction to Haskell}
\author[Franti\v{s}ek Farka]{Franti\v{s}ek Farka}
\date{}
\usetikzlibrary{arrows}

\newcommand{\boxtt}[1]{\mbox{\scriptsize\texttt{#1}}}\newcommand{\cU}{\mathcal{U}}


\begin{document}
\BackgroundPicture{fondo1.png}
\begin{frame}
\titlepage
\end{frame}


\BackgroundPicture{fondo.png}
\begin{frame}
    \frametitle{Syllabus of this part of the AC21007 module}

    Lectures
    \begin{itemize}
        \item Lecture 1: An introduction to Haskell
        \item Lecture 2: List functions, function polymorphism
        \item Lecture 3: Folds, tail recursion
        \item Lecture 4: Data types
        \item Lecture 5: Type classes
        \item Lecture 6: Sorting algorithms in Haskell
        \item Lecture 7: AI Algorithms: BFS, DFS, ...
        \item Lecture 8: tbd
    \end{itemize}

    Labs 
    \begin{itemize}
        \item each week, half a slot
        \item please use the Lab 2
        \item weekly lab exercises, a final assignment
    \end{itemize}

\end{frame}


\begin{frame}
    \frametitle{Resources}
 
    \begin{itemize}

    \item Books:

        \beamertemplatebookbibitems
        \begin{thebibliography}{99}
            \bibitem{haskell}{Thompson, Simon. Haskell: The craft of functional programming(3rd Edition)\\ 
            }
            \bibitem{haskell}{Mena, Alejandro Serrano.                   
                    Beginning Haskell: A Project-Based Approach\\
            }
        \end{thebibliography}

    \item Free online resources:
        \begin{thebibliography}{99}

            \bibitem{lyah}{Learn You a Haskell for Great Good!\\ 
                \url{http://learnyouahaskell.com/}
            }
            \bibitem{rwh}{
                    Real World Haskell\\ 
                \url{http://book.realworldhaskell.org/}
            }
        \end{thebibliography}
    \end{itemize}

    
    \end{frame}


    \begin{frame}
    \frametitle{History of Haskell}

    \begin{itemize}
    \item Named after the logician Haskell Curry
    \item Conference on Functional Programming Languages and Computer Architecture in Portland, Oregon. 1987
    \item Common language for \textit{lazy functional programming languages} research
    \item Matured a lot over the past 29 years
    \item Several standards: Haskell98, Haskell2010
    \end{itemize}

    \pause

    \begin{itemize}
        \item De facto implementation: Glasgow Haskell Compiler
            (GHC)\footnote{Current: The Glorious Glasgow Haskell
                Compilation System, version 7.10.2}
    \item New development of the language via GHC language extensions
    \end{itemize}

    \end{frame}

    \begin{frame}
    \frametitle{Notable Features of Haskell}

    {\bf Haskell} \textipa{/"h{\ae}sk@l/} is a standardised, general-purpose purely functional
    programming language, with non-strict semantics and strong static
    typing.\footnote{\url{https://en.wikipedia.org/wiki/Haskell_(programming_language)}}


    \begin{itemize}
    \item<2-> purely functional
    
    \item<3-> non-strict (also lazy) semantics

    \item<4-> (strong) static typing
    
    \end{itemize}

    \end{frame}


    \begin{frame}
    \frametitle{Programming Language Paradigms}
    
    \begin{itemize}
    \item<1-> Procedural languages: Pascal, C, Python, ...
    \item<2-> Object Oriented languages: C++, Java, C\#, Python... 
    \item<3-> Functional languages: Haskell, ML, OCaml, Lisp,...
    \item<4-> Functional and Object Oriented languages: Scala,...  
    \item<5-> Declarative-Logic languages: Prolog, ...
    \item<6-> and many others: Erlang, Go,...
    \end{itemize}

    \end{frame}


\begin{frame}[fragile]
    \frametitle{Imperative Style: C}
    Power function: %$b^0 = 1$, $b^{n} = b*b^{n-1}$ if $n>0$.
    $ b^n = \begin{cases} 
        1 & n =  0 \\
        b * b ^ {n-1} & n > 0
        \end{cases}
    $

    \pause

    \vskip 1em
    An implementation in C:

    \begin{verbatim}
int power(int b, int n)
{
    int p = 1;

    for ( ; 0 < n; --n)
       p = p * b;
    return p;
}
\end{verbatim}

    \pause
    {\tt power(7,3) $==>$ 343}
 
\end{frame}

\begin{frame}[fragile]
    \frametitle{Imperative Style: C}
    %Power function: %$b^0 = 1$, $b^{n} = b*b^{n-1}$ if $n>0$.
    %$ b^n = \begin{cases} 
    %    1 & n =  0 \\
    %    b * b ^ {n-1} & n > 0
    %    \end{cases}
    %$

    %\vskip 1em
    A bit more verbose implementation in C:

    \pause

    \begin{verbatim}
int power(b, n)
    int b;
    int n;
{
    int p;
    p = 1;

forcycle:
    if (! (0 < n )) goto endfor; 
        p = p * b;
        --n;
    goto forcycle;
endfor:

    return p;
}
\end{verbatim}
 
\end{frame}


\begin{frame}[fragile]
 \frametitle{Functional Style: Haskell}
    Power function: %$b^0 = 1$, $b^{n} = b*b^{n-1}$ if $n>0$.
    $ b^n = \begin{cases} 
        1 & n =  0 \\
        b * b ^ {n-1} & n > 0
        \end{cases}
    $

    \pause
    \vskip 1em

A Haskell implementation:

\begin{verbatim}
power :: Int -> Int -> Int
power b 0 = 1
power b n = b * (power b (n - 1))
\end{verbatim}

\pause
    {\tt \small
        power 7 3 \\ 
        \pause $==>$ 7 * (power 7 (3 - 1)) \\
        \pause $==>$ 7 * (power 7 2) \\ 
        \pause $==>$ 7 * (7 * (power 7 (2 - 1))) \\ 
        \pause $==>$ 7 * (7 * (power 7 1)) \\ 
        \pause $==>$ 7 * (7 * (7 * (power 7 (1 - 1)))) \\
        \pause $==>$ 7 * (7 * (7 * (power 7 0))) \\
        \pause $==>$ 7 * (7 * (7 * (1)) \\
        \pause $==>$ 7 * (7 * (7)) \\ 
        \pause $==>$ 7 * (49) \\
        \pause $==>$ 343
    }


\end{frame}

\begin{frame}[fragile]
    \frametitle{Functional Style: Haskell (cont.)}
    Logical negation: %$b^0 = 1$, $b^{n} = b*b^{n-1}$ if $n>0$.
    $ \neg p = \begin{cases} 
        False & p =  True \\
        True  & p = False
        \end{cases}
    $

    \vskip 1em

A Haskell implementation:

\begin{alltt}
not :: Bool -> Bool
not True  = False
not False = True
\end{alltt}

\end{frame}

\begin{frame}[fragile]
  \frametitle{Purity}
\begin{verbatim}  
int counter = 0;
int power(int b, int n)
{
    if (counter > 5) return 42;
    counter++;

    int p = 1;
    for (; 0 < n; --n)
       p = p * b;
    return p;
}  
\end{verbatim}  

\pause

\begin{alltt}
power :: Int -> Int -> Int
power b 0 = 1
power b n = b * (power b (n-1))
\end{alltt}

\pause

\begin{itemize}
    \item Both power functions have similar type
    \item However, Haskell's power function is \emph{pure} - has no side effects
\end{itemize}
\end{frame}  



\begin{frame}
    \frametitle{Summary}

  Imperative Style:
  \begin{itemize}
    \item Close to machine   
    \item Programming by iteration and by modifications of state (for-loops, updates of variables)
    \item Side-effects
  \end{itemize}

  Functional Style:
  \begin{itemize}
    \item Close to problem description
    \item Program by recursion (no for-loops, no updates of variables)
    \item No side-effects

  \end{itemize}
\end{frame}


\begin{frame}
    \frametitle{Development environment}

    \begin{itemize}
        \item A text editor of your choice

        \item Haskell Platform - \url{https://www.haskell.org/platform/}
            \begin{itemize}
                \item The Glasgow Haskell Compiler
                \item The Cabal build system
                \item 35 core \& widely-used packages (libraries)
            \end{itemize}
    \end{itemize}

\end{frame}


\begin{frame}
    \frametitle{Development environment (cont.)}

    \begin{itemize}
        \item A word about GHC
            \begin{itemize}
                \item a compiler {\tt ghc}
                \item an interpreter {\tt ghci}
                \item a limited set of standard packages ({\tt base}, \dots)
            \end{itemize}

        \item A word about Cabal
            \begin{itemize}
                \item a package manager {\tt cabal}
                \item used both for installing new packages and project
                    management
                \item uses online repository Hackage
            \end{itemize}

        \item A word about Hackage (\url{https://hackage.haskell.org/})
            \begin{itemize}
                \item community's central package archive
                \item contains generated documentation
                \item search engine Hoogle
                    (\url{https://www.haskell.org/hoogle/})
            \end{itemize}


    \end{itemize}

\end{frame}

\begin{frame}[fragile]
    \frametitle{Hello World!}

    \begin{itemize}
        \item Because every language tutorial has it, and
        \item to show you few other basics.
    \end{itemize}

    \pause

    a file \emph{Main.hs}:
    \begin{verbatim}
module Main where

{-  Simple function to create a hello 
    message. -}
hello :: String -> String
hello s = "\n\tHello " ++ s

-- Entry point of a program
main :: IO ()
main = putStrLn (hello "World")
    \end{verbatim}
\end{frame}

\begin{frame}
    \frametitle{Next time}

    \begin{itemize}
        \item one extra lecture, Wednesday the the 20th of January, 11-12AM,
            { \color{red}{Wolfson LT, QMB} }

        \item lists and list functions
        \item non-strict (lazy) semantics
        \item function polymorphism

    \end{itemize}

\end{frame}

%%%%%%%%%%%%% 
% unused

\begin{frame}
  \frametitle{GHCi demo}
  \url{http://www.tutorialspoint.com/compile_haskell_online.php}
\end{frame}

%\begin{frame}[fragile]
%  \frametitle{Types and Purity}
%\begin{verbatim}  
%int power(int b, int n)
%{
%    int p = 1;
%    for (; 0 < n; --n)
%       p = p * b;
%    erase-hard-drive();       
%    return p;
%}  
%\end{verbatim}  
%\begin{alltt}
%power :: Int -> Int -> Int
%power b 0 = 1
%power b n = b * (power b (n-1))
%\end{alltt}
%
%\begin{itemize}
%\item Both power functions have similar type
%\item However, Haskell's power function can't erase a hard-drive \dots
%\end{itemize}
%\end{frame}  

\end{document}
