\documentclass[final,handout]{beamer}
%\documentclass[]{beamer}
\definecolor{green}{rgb}{0,0.6,0}
\definecolor{blue}{rgb}{0,0, 0.6}
\definecolor{red}{rgb}{0.6,0, 0}
\definecolor{gray}{rgb}{0.6,0.6, 0.6}
%\setbeamertemplate{headline}[verdeuni]
%\setbeamercovered{highly dinamic}
%\usepackage{eso-pic}

\usepackage[utf8]{inputenc}
\usepackage[english]{babel}
\usepackage{color}
\usepackage{url}
\usepackage{hyperref}
\usepackage{fancyvrb}
\usepackage{tikz}
\usepackage{alltt}
\usepackage{etex, xy}
\usepackage{tipa}

\xyoption{all} %para los diagramas
\usepackage{cancel}
\usepackage{verbatim}
%\usepackage{slashbox}
\usetikzlibrary{decorations.pathreplacing,shapes.arrows}
\newcommand\BackgroundPicture[1]{%
  \setbeamertemplate{background}{%
   \parbox[c][\paperheight]{\paperwidth}{%
      \vfill \hfill
\includegraphics[width=1\paperwidth,height=0.9\paperheight]{#1}
        \hfill \vfill
     }}}

\usepackage{color}
\usepackage{listings}
\lstset{
    language=haskell
}


\newcommand{\rrdc}{\mbox{\,\(\Rightarrow\hspace{-9pt}\Rightarrow\)\,}} % Right reduction
\newcommand{\lrdc}{\mbox{\,\(\Leftarrow\hspace{-9pt}\Leftarrow\)\,}}% Left reduction
\newcommand{\lrrdc}{\mbox{\,\(\Leftarrow\hspace{-9pt}\Leftarrow\hspace{-5pt}\Rightarrow\hspace{-9pt}\Rightarrow\)\,}} % Equivalence
\DeclareMathOperator{\id}{Id}
\newcommand{\sCoq}{{\sc Coq}}
\newcommand{\SSReflect}{{\sc SSReflect}}
\newcommand{\Zset}{\mathbb{Z}}
\newcommand{\Bset}{\mathbb{Z}_2}


\newtheorem{Thm}{Theorem}
\newtheorem{Prop}{Proposition}
\newtheorem{Lem}{Lemma}
\newtheorem{Cor}{Corollary}
\newtheorem{Defn}{Definition}
\newtheorem{Goal}{Goal}
\newtheorem{GGoal}{General Goal}
\newtheorem{Alg}{Algorithm}
\newcommand{\inp}{{\itshape Input: }}
\newcommand{\outp}{{\itshape Output: }}

\renewcommand{\thefootnote}{$\ast$}
\title[Haskell Lecture 7]{AC21007: Haskell Lecture 7\\
    Quick Sort,
    %Merge Sort,
    Monadic IO
}
\author[Franti\v{s}ek Farka]{Franti\v{s}ek Farka}
\date{}
\usetikzlibrary{arrows}

\newcommand{\boxtt}[1]{\mbox{\scriptsize\texttt{#1}}}\newcommand{\cU}{\mathcal{U}}


\begin{document}
\BackgroundPicture{fondo1.png}
\begin{frame}
\titlepage
\end{frame}

\begin{frame}[fragile]
    \frametitle{Recapitulation}

    \begin{itemize}
        \item Tail recursion
            \begin{itemize}
                \item Sum
                \item Fibonacci numbers
                \item Tail recursion and folds
            \end{itemize}
        \item Algebraic data types
        \item (Light introduction to) Typeclasses
  \end{itemize}
\end{frame}  


\begin{frame}[fragile]
    \frametitle{Quick Sort: Intuition}

    \begin{enumerate}
        \item Choose an element in a list as ``pivot''
        \item Move all the elements larger than pivot to its right.
        \item Move all the elements smaller than pivot to its left.
        \item Recursively sort elements on left and on right of the pivot
    \end{enumerate}
\end{frame}

\begin{frame}[fragile]
    \frametitle{Quick Sort in Haskell}

    \begin{itemize}
        \item Quick Sort has two nice aspects:
            \begin{itemize}
                \item Divide and Conquer
                \item In-place sort
            \end{itemize}
        \item<2-> In-place sort like quick sort requires mutable arrays and
             mutable variables.
        \item<3-> To get pure version of quick sort, we need to forget about
                swapping, indexing, mutation.
        \item<4-> Think in terms of creating new list based on input list.
    \end{itemize}
\end{frame}

\begin{frame}[fragile]
    \frametitle{Quick Sort in Haskell (cont.)}

    \begin{itemize}
        \item How to pick a pivot? \pause Take the first element.
        \item<3-> Sort a list:

            \begin{verbatim}
quickSort [] = []
quickSort (x:xs) =
        let (left, right) = partition xs x
        in quickSort left ++ [x] ++ quickSort right
    where
        partition []     _  = ([], [])
        partition (y:ys) z  =
            let (vs, ws) = partition ys
            in if (y < z) 
                    then (y:vs, ws)
                    else (vs, y:ws)
            \end{verbatim}
    \end{itemize}
\end{frame}

\begin{frame}[fragile]
    \frametitle{Quick Sort in Haskell (cont.)}

    \begin{itemize}
        \item Quick Sort has two nice aspects:
            \begin{itemize}
                \item Divide and Conquer
                \item In-place sort
            \end{itemize}
        \item Our version only demonstrate the divide and conquer part.
        \item Worst case time complexity: $\mathcal{O}(n^2)$
        \item Average time complexity: $\mathcal{O}(n \log n)$
    \end{itemize}
\end{frame}




%\begin{frame}[fragile]
%    \frametitle{Merge Sort}
%
%    \begin{itemize}
%        \item unfold - fold?
%    \end{itemize}
%
%\end{frame}

\begin{frame}[fragile]
    \frametitle{Syntactic Intermezzo: case expression}

    \begin{itemize}
        \item We saw ADTs
        \item How do we inspect values of ADTs?
            \begin{itemize}
                \item Pattern matching in function definition
                \item \texttt{case} expression
            \end{itemize}
        \item<2-> Syntax of \texttt{case} expression:

\begin{alltt}
    case <expr> of
        <pat\(\sb{1}\)>      ->  <expr\(\sb{1}\)>
        \dots
        <pat\(\sb{n}\)>      ->  <expr\(\sb{n}\)>
\end{alltt}
        $<expr_1>$ to $<expr_n>$ are of some type $a$, the \texttt{case}
        expression has a value of the type $a$, e.g.:
  
\begin{verbatim}
    case (safeHead someList) of
        Nothing -> "No head"
        Just h  -> "The head is: " ++ show h
\end{verbatim}
    

    \end{itemize}

\end{frame}

\begin{frame}[fragile]
    \frametitle{Maybe as a monadic computation}
    
    \begin{itemize}
        \item We saw the \texttt{Maybe} data type
        \item We saw that we can use it to enrich a range of a function (e.~g. to
            make a partial function total):
\pause
    \begin{verbatim}
        head :: [a] -> a
        head []     = error "Empty list"
        head (x:_)  = x
    \end{verbatim}

    vs.

    \begin{verbatim}
        safeHead :: [a] -> Maybe a
        safeHead []     = Nothing
        safeHead (x:_)  = Just x
    \end{verbatim}

        \item<3-> We will call \texttt{Maybe} is such a situation a \emph{context}
            of a computation 
    \end{itemize}

\end{frame}

\begin{frame}[fragile]
    \frametitle{Maybe as a monadic computation (cont.)}
    
    \begin{itemize}
        \item Lets see how composable this approach is:
    \begin{verbatim}
    sqrtHead :: [Float] -> Float
    sqrtHead xs = sqrt (head xs)
    \end{verbatim}
    \begin{itemize}
        \item \texttt{head} fails on an empty list
        \item \texttt{sqrt} fails on a negative number
    \end{itemize}
    
        \item<2-> We already have \texttt{safeHead}, can we provide \texttt{safeSqrt}?

    \begin{verbatim}
    safeSqrt :: Float -> Maybe Float
    safeSqrt a = if a < 0 
                    then Nothing
                    else Just (sqrt a)
    \end{verbatim}

        \item<3-> Let's compose these two into \texttt{safeSqrtHead} \dots
    \end{itemize}

\end{frame}

\begin{frame}[fragile]
    \frametitle{Maybe as a monadic computation (cont.)}
    
    \begin{itemize}
        \item Lets see how composable this approach is:
    \begin{verbatim}
    sqrtHead :: [Float] -> Float
    sqrtHead xs = sqrt (head xs)

    safeSqrtHead :: [Float] -> Maybe Float
    safeSqrtHead xs = case safeHead xs of
        Nothing     -> Nothing
        Just x      -> safeSqrt x
    \end{verbatim}

    
    ~\only<2->{\dots the explicit \texttt{case} is verbose}

        \item Note the type signatures:

    \begin{verbatim}
    safeHead :: [a] -> Maybe a
    safeSqrt :: Float -> Maybe Float
    \end{verbatim}
    \end{itemize}

\end{frame}

\begin{frame}[fragile]
    \frametitle{Maybe as a monadic computation (cont.)}
    
    \begin{itemize}
        \item Lets see how composable this approach is:
    {\tt \\~\\
    safeSqrtHead :: [Float] -> Maybe Float\\
    safeSqrtHead xs = safeHead xs `bind` safeSqrt\\
    ~\\
    \only<1-2>{bind :: Maybe Float -> (Float -> Maybe Float)\\ 
    ~~~~-> Maybe Float}
    \only<3->{~\\bind :: Maybe a -> (a -> Maybe b) -> Maybe b}\\
    bind mval func   = case mval of\\
    ~~~~~~~~Nothing~~~~~-> Nothing\\
    ~~~~~~~~Just val~~~~-> func val\\
    ~\\
    }
    ~\only<2->{\dots What is the most generic type of \texttt{bind}?}

        \item Note the type signatures:

    \begin{verbatim}
    safeHead :: [a] -> Maybe a
    safeSqrt :: Float -> Maybe Float
    \end{verbatim}
    \end{itemize}

\end{frame}

\begin{frame}[fragile]
    \frametitle{\texttt{Monad} typeclass}
    
    \begin{itemize}
        \item We can abstract this technique over different data types using
            a typeclass (think of data types being ``bindable'' in the same way
            as being ``orderable'' and \texttt{Ord} typeclass)
        \item<2-> The \texttt{Monad} t.c. as an interface for binding computations:
{\small
    \begin{verbatim}
class Monad m where
    -- an operator instead of our `bind`
    (>>=) :: m a -> (a -> m b) -> m b
    return :: a -> m a

instance Monad Maybe where
    Nothing     >>= _   = Nothing
    (Just a)    >>= f   = f a
    return a            = Just a
    \end{verbatim}
}
        The \texttt{return} fnct to embed a pure value
        into a context

        \item<3-> And our previous use case:
{\small
\begin{verbatim}
    safeSqrtHead xs = safeHead xs >>= safeSqrt 
    sqrtOfTwo = return 2 >>= safeSqrt
\end{verbatim}
}
    \end{itemize}

\end{frame}

\begin{frame}[fragile]
    \frametitle{Unit Data type - ()}

    \begin{itemize}
        \item In Haskell all functions return a value
        \item Sometimes, we are not interested in the actual  value
        \item There is a data type for this --- \texttt{()} (unit) --- that has
            a single constructor---also \texttt{()}.
    \end{itemize}
\end{frame}



\begin{frame}[fragile]
    \frametitle{Monadic IO}

    \begin{itemize}
        \item In Haskell all IO happens in a context of type \texttt{IO a}
        \item \texttt{IO} encapsulates a state of the real world, you cannot
            construct or inspect values of this type directly

        \item<2-> There are functions that take or return \texttt{IO} values:

            \begin{itemize}
                \item \texttt{putStr, putStrLn :: String -> IO ()}
                \item \texttt{getLine :: IO String}
            \end{itemize}

        \item<3-> And there is a \texttt{Monad IO} instance---\texttt{IO} computation can be
            sequenced using bind ($>>=$), a pure value can be injected into an
            \texttt{IO} context using \texttt{return}:

        \begin{verbatim}
helloYou =  getLine >>= \x -> 
            putStrLn ("Hello " ++ x)
        \end{verbatim}

        \item<4-> We also say that there is an \emph{effect}, which is performed in
            a monadic context (in general, not only \texttt{IO}).
    \end{itemize}
\end{frame}

\begin{frame}[fragile]
    \frametitle{Do Notation}

    \begin{itemize}
        \item There is a syntax for monadic computations --- \texttt{do}
            notation
        \item<2-> We call a single call to a function that returns monadic value an
            \emph{action}. We either bind a value \textbf{in} this context to a
            variable:
            \begin{alltt}
    var\(\sb{n}\) <- action\(\sb{n}\)
            \end{alltt}
            \vskip -1em
\pause
            or we ignore this value (we are interested only in the effect)
            \begin{alltt}
    action\(\sb{n}\)
            \end{alltt}
            \vskip -1em
\pause
            and we sequence such actions in a block, while using bound
            variables as arguments of other actions (following the action that
            binds the variable):
 
            \begin{alltt}
    do
        var\(\sb{1}\) <- action\(\sb{1}\)
        var\(\sb{2}\) <- action\(\sb{2}\) 
        \dots
        action\(\sb{n}\) var\(\sb{i}\) var\(\sb{j}\)
            \end{alltt}
            \vskip -1em
\pause
        the result of a \texttt{do} block is the result of last action (this
        action must not be a binding of a variable)


            

    \end{itemize}
\end{frame}

\begin{frame}[fragile]
    \frametitle{IO -- A Simple Example}

    \begin{itemize}
        \item A simple example of IO:
            \begin{lstlisting}
-- | Prompts a user for a number
getNumber :: String -> IO Int
getNumber username = do
        putStrLn ("Hello " ++ username ++ "!" 
          ++ "Choose your favourite number:")
        x <- getLine
        putStrLn "Thank you!"
        return (read x)
            \end{lstlisting}
    \end{itemize}
\end{frame}

\begin{frame}[fragile]
    \frametitle{An overview of IO functions}
\small
    \begin{itemize}
        \item[putChar] \texttt{:: Char -> IO ()} \\
            Write a character to the standard output device
        \item[putStr] \texttt{:: String -> IO ()} \\
            Write a string to the standard output device
        \item[putStrLn] \texttt{:: String -> IO ()} \\
            The same as putStr, but adds a newline character.
        \item[getChar] \texttt{:: IO Char} \\
            Read a character from the standard input device
        \item[getLine] \texttt{:: IO String} \\
            Read a line from the standard input device 
        \item[type] \texttt{FilePath = String}
        \item[readFile]\texttt{:: FilePath -> IO String} \\
            Returns the contents of the file as a string.

        \item[writeFile] \texttt{:: FilePath -> String -> IO ()} \\
            Writes a string to a file.

        \item[getArgs] \texttt{:: IO [String]}
            Returns a list of the program's command line arguments
            (in \texttt{System.Environment})
    \end{itemize}
\end{frame}

\begin{frame}[fragile]
    \frametitle{IO -- A More Complex Example}

    \begin{itemize}
        \item Read file name from the input, sort it, write it to the output
\small
\begin{lstlisting}
import System.Environment (getArgs)

main = do
    args <- getArgs
    if null args
        then print "Provide a filename"
        else do
            fileCnt <- readFile (head args)
            let cnt :: [Int]
                cnt = map read (lines fileCnt)
            putStrLn (show (quickSort cnt))
            writeFile
                (mkName (head args)) 
                ("#sorted: " ++ show (length cnt))
    where
        mkName name = takeWhile (/= '.') name
            ++ ".out"
\end{lstlisting}
    
        \item For more detailed description of functions use \emph{Hoogle}


    \end{itemize}
\end{frame}

\begin{frame}
    \frametitle{Last lecture}

    \begin{itemize}
        \item This was the last lecture
        
        \item Thank you for you patience

        \item Please send me a feedback or any comments to
            \url{frantisek@farka.eu}

    \end{itemize}

\end{frame}


\end{document}

\begin{frame}[fragile]
    \frametitle{Foo}

    \begin{itemize}
        \item
    \end{itemize}
\end{frame}

