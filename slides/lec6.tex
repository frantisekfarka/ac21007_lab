\documentclass[final,handout]{beamer}
%\documentclass[]{beamer}
\definecolor{green}{rgb}{0,0.6,0}
\definecolor{blue}{rgb}{0,0, 0.6}
\definecolor{red}{rgb}{0.6,0, 0}
\definecolor{gray}{rgb}{0.6,0.6, 0.6}
%\setbeamertemplate{headline}[verdeuni]
%\setbeamercovered{highly dinamic}
%\usepackage{eso-pic}

\usepackage[utf8]{inputenc}
\usepackage[english]{babel}
\usepackage{color}
\usepackage{url}
\usepackage{hyperref}
\usepackage{fancyvrb}
\usepackage{tikz}
\usepackage{alltt}
\usepackage{etex, xy}
\usepackage{tipa}

\xyoption{all} %para los diagramas
\usepackage{cancel}
\usepackage{verbatim}
%\usepackage{slashbox}
\usetikzlibrary{decorations.pathreplacing,shapes.arrows}
\newcommand\BackgroundPicture[1]{%
  \setbeamertemplate{background}{%
   \parbox[c][\paperheight]{\paperwidth}{%
      \vfill \hfill
\includegraphics[width=1\paperwidth,height=0.9\paperheight]{#1}
        \hfill \vfill
     }}}

\usepackage{color}
\usepackage{listings}
\lstset{
    language=haskell
}


\newcommand{\rrdc}{\mbox{\,\(\Rightarrow\hspace{-9pt}\Rightarrow\)\,}} % Right reduction
\newcommand{\lrdc}{\mbox{\,\(\Leftarrow\hspace{-9pt}\Leftarrow\)\,}}% Left reduction
\newcommand{\lrrdc}{\mbox{\,\(\Leftarrow\hspace{-9pt}\Leftarrow\hspace{-5pt}\Rightarrow\hspace{-9pt}\Rightarrow\)\,}} % Equivalence
\DeclareMathOperator{\id}{Id}
\newcommand{\sCoq}{{\sc Coq}}
\newcommand{\SSReflect}{{\sc SSReflect}}
\newcommand{\Zset}{\mathbb{Z}}
\newcommand{\Bset}{\mathbb{Z}_2}


\newtheorem{Thm}{Theorem}
\newtheorem{Prop}{Proposition}
\newtheorem{Lem}{Lemma}
\newtheorem{Cor}{Corollary}
\newtheorem{Defn}{Definition}
\newtheorem{Goal}{Goal}
\newtheorem{GGoal}{General Goal}
\newtheorem{Alg}{Algorithm}
\newcommand{\inp}{{\itshape Input: }}
\newcommand{\outp}{{\itshape Output: }}

\renewcommand{\thefootnote}{$\ast$}
\title[Haskell Lecture 6]{AC21007: Haskell Lecture 6\\
    Tail Recursion, Algebraic Data Types, Type Classes
}
\author[Franti\v{s}ek Farka]{Franti\v{s}ek Farka}
\date{}
\usetikzlibrary{arrows}

\newcommand{\boxtt}[1]{\mbox{\scriptsize\texttt{#1}}}\newcommand{\cU}{\mathcal{U}}


\begin{document}
\BackgroundPicture{fondo1.png}
\begin{frame}
\titlepage
\end{frame}

\begin{frame}[fragile]
    \frametitle{Recapitulation}

    \begin{itemize}
        \item Sorting algorithms
            \begin{itemize}
                \item Selection Sort
                \item Insertion Sort
                \item Bubble Sort
            \end{itemize}
  \end{itemize}
\end{frame}  


\begin{frame}[fragile]
    \frametitle{Tail recursion}

    \begin{itemize}
        \item A recursive function is tail recursive iff the final result of the
            recursive call is the final result of the function itself
        \item I.e. the outermost function applied in an RHS expression.

        \item<2-> Non-tail recursive \texttt{sum}:
        \begin{lstlisting}
    sum :: [Int] -> Int
    sum []      = 0
    sum (x:xs)  = x + (sum xs)
        \end{lstlisting}

    \item<3-> A tail-recursive version - we use an additional accumulator \texttt{acc}:
            \begin{lstlisting}
    sumAux :: [Int] -> Int -> Int
    sumAux []       acc = acc {- + 0 -}
    sumAux (x:xs)   acc = sumAux xs (acc + x)

    sum :: [Int] -> Int
    sum xs = sumAux xs 0
            \end{lstlisting}

    \end{itemize}
\end{frame}

\begin{frame}[fragile]
    \frametitle{Tail recursion - Fibonacci numbers}

    \begin{itemize}

        \item Fibonacci numbers: $F_n = 
            \begin{cases}
                = 0 & n = 0 \\
                = 1 & n = 1 \\
                = F_{n-1} + F_{n-2} & \text{otherwise} 
            \end{cases}$

            ~\\
            $0, 1, 1, 2, 3, 5, 8, 13, \dots$

        \item<2-> Haskell implementation is straightforward:

            \begin{lstlisting}
    fib :: Integer -> Integer
    fib 0 = 0
    fib 1 = 1
    fib n = fib (n - 1) + fib (n - 2)
            \end{lstlisting}

        \item<3-> Can we turn this into a tail-recursive function?

    \end{itemize}

\end{frame}

\begin{frame}[fragile]
    \frametitle{Tail recursion - Fibonacci numbers (cont.)}

    \begin{itemize}
        \item Observation: recursive step performs two recursive calls
        \item<2-> \dots in \texttt{sum} it performs one r.c. and uses one
            \texttt{acc} \dots
        \item<3-> \dots we are going to use two intermediate values!

        \item<4-> An implementation:
            \begin{lstlisting}
    fibHelper :: Int -> Int -> Int -> Int
    fibHelper 0 val1    val2    = val1
    fibHelper 1 val1    val2    = val2
    fibHelper n val1    val2    =
        fibHelper (n - 1) val2 (val1 + val2)

    fib :: Int -> Int
    fib n = fibHelper n 0 1
            \end{lstlisting}

    \end{itemize}
\end{frame}

\begin{frame}[fragile]
    \frametitle{Tail recursion and folds}

    \begin{itemize}
        \item We already saw folds - ``schemes" of recursive functions

        \item We know that e.g. \texttt{sum} can be expressed as a fold:
            \begin{lstlisting}
    sum :: [Int] -> Int
    sum xs = foldl (+) 0 xs
            \end{lstlisting}
            or
            \begin{lstlisting}
    sum' :: [Int] -> Int
    sum' = foldr (+) 0
            \end{lstlisting}

        \item Is either of these tail-recursive? 
    \end{itemize}
\end{frame}

\begin{frame}[fragile]
    \frametitle{Tail recursion and folds (cont.)}

    \begin{itemize}
        \item recall recursive steps of \texttt{foldr} and \texttt{foldl}:

            \begin{lstlisting}
    foldr f z (x:xs) = f x (foldr f z (x:xs))

    ...

    foldl f z (x:xs) = let z' = z `f` x 
                       in foldl f z' xs
            \end{lstlisting}

        \item<2-> \texttt{foldr} is \emph{not} tail-recursive but \texttt{foldl} is
            tail recursive!
    \end{itemize}
\end{frame}

\begin{frame}[fragile]
    \frametitle{Algebraic Data Types}

    \begin{itemize}
        \item We define our own data types by stating data type name and it's
            \emph{constructors} -- both identifiers types and constructors begin
            with an uppercase letter

        \item<2-> We already know \texttt{Bool}:
            \begin{lstlisting}
    data Bool = False | True
            \end{lstlisting}

            \texttt{Bool} is a type with two constructors: \texttt{True} and
            \texttt{False}.

        \item<3-> Similarly we can define e.g.:

            \begin{lstlisting}
    data Suits  = Spades 
                | Hearts
                | Diamonds
                | Clubs
            \end{lstlisting}
    \end{itemize}
\end{frame}

\begin{frame}[fragile]
    \frametitle{Algebraic Data Types (cont.)}

    \begin{itemize}
        \item<1-> We also saw tuples and a constructor \texttt{(,)} -- e.g.:\\
            \begin{lstlisting}
    (1, 'c') :: (Int, Char)
            \end{lstlisting}

        \item<2-> Constructors may contain \emph{fields} of certain type, e.g.:
            \begin{lstlisting}
    data MyPair = MyPair Int Char
            \end{lstlisting}
            Note that the name of a type  and a name of it's constructor can be
            the same. A value of type \texttt{MyPair}:\\
            \begin{lstlisting}
    myPairVal :: MyPair
    myPairVal = MyPair 1 'c' 
            \end{lstlisting}

        \item<3-> Values of algebraic data types are constructed in the same way
            as values of lists and tuples.
        
    \end{itemize}
\end{frame}

\begin{frame}[fragile]
    \frametitle{Algebraic Data Types (cont.)}

    \begin{itemize}
        \item We can also pattern-match on data type values in function definitions and let-bindings
            in the very same way as with lists and tuples:

            \begin{lstlisting}
    incMyPair :: MyPair -> MyPair
    incMyPair (MyPair i c) = MyPair (i + 1) c
            \end{lstlisting}

        \item<2-> \dots but tuple type is more general -- \texttt{(a, b)} for any
            types \texttt{a} and \texttt{b}
    \end{itemize}
\end{frame}

\begin{frame}[fragile]
    \frametitle{Algebraic Data Types (cont.)}

    \begin{itemize}
        \item Data types may be polymorphic in fields of constructors,
            
            \begin{lstlisting}
    data Pair a b = Pair a b
            \end{lstlisting}

            we can specify type variables after the name of type and use them as
            types of constructor fields.

        \item<2-> And we can combine all of the above:
            \begin{lstlisting}
    data CrazyType a b c
        = NoParamCtor
        | MonoMorphicCtor1 Int 
        | MonoCtor2 String [Char] Bool
        | MonoCtor3 MyPair
        | PolyCtor1 a b
        | PolyCtor2 a Int
        | PolyCtor3 (Pair c Int)
        | PolyCtor4 (c -> a, Int)
            \end{lstlisting}

        \item We call these data types \emph{Algebraic Data Types} (ADT's)
    \end{itemize}
\end{frame}

\begin{frame}[fragile]
    \frametitle{Some old ADT's \dots}

    \begin{itemize}
        \item<1-> The list type is just an ordinary type, the only special thing is
            syntactic sugar for ``\texttt{[]}'' and ``\texttt{(:)}'':

            \begin{lstlisting}
    data List a = Nil | Cons a (List a)

    length' :: List a -> Int
    length' Nil         = 0
    length' (Cons _ xs) = 1 + length' xs
            \end{lstlisting}

        \item<2-> And the same for tuples, as we already saw:

            \begin{lstlisting}
    data Pair a b = Pair a b

    fst' :: Pair a b -> a
    fst' (Pair x _) = x

    snd' :: Pair a b -> b
    snd' (Pair _ y) = y
            \end{lstlisting}

    \end{itemize}
\end{frame}

\begin{frame}[fragile]
    \frametitle{\dots and some new}

    \begin{itemize}
        \item<1-> Sometimes, we need an extra value:
        \begin{lstlisting}
    data Maybe a = Nothing | Just a

    safeHead :: [a] -> Maybe a
    safeHead []     = Nothing
    safeHead (x:_)  = Just x
        \end{lstlisting}

    \item<2-> ADT representing binary trees (values are only in leafs):

        \begin{lstlisting}
    data BinTree a
        = Leaf a 
        | Node (BinTree a) (BinTree a)

    myTree :: BinTree Char
    myTree = Node (Leaf 'a') (Leaf 'b')
        \end{lstlisting}

    \end{itemize}
\end{frame}

\begin{frame}[fragile]
    \frametitle{Type Classes}

    \begin{itemize}
        \item So far we saw monomorphic functions, e.g. \texttt{neg},
            \texttt{and}, and polymorphic functions, e.g. \texttt{fst},
            \texttt{head}.

        \item<2-> What if we want a function, that is polymorphic only for some types
            (ad-hoc polymorphism), e.g. \texttt{sort} for \texttt{Int},
            \texttt{Integer}, and \texttt{Float}?
            
{\tt
    ~\\
~~~~sort :: \only<3->{Ord a =>} [a] -> [a] \\
~~~~sort = ...\only<3->{~~\{- (<=) for a-values  -\}}\\
    ~\\
}

            We need to constrain type variable $a$ to types, that can be
            ordered. 

        \item<4-> \texttt{Ord a} is a type class \emph{constraint}.


    \end{itemize}
\end{frame}

\begin{frame}[fragile]
    \frametitle{Type Classes (cont.)}

    \begin{itemize}
        \item We can define a \emph{class} of types and specify which functions
            (called class \emph{methods}) are
            available for types of this class (i.e. type class behaves as an
            interface), e.g.:

            \begin{lstlisting}
    class Ord a where
        (<=) :: a -> a -> Bool
            \end{lstlisting}

        \item<2-> We can specify, that a type is and \emph{instance} of a class --
            we provide an implementation of class functions for this type:

           \begin{lstlisting}
    instance Ord Int where
        x <= y = primitiveIntComparison x y

    instance Ord Float where
        x <= y = primitiveFloatComparison x y
           \end{lstlisting}

    \end{itemize}
\end{frame}

\begin{frame}[fragile]
    \frametitle{Type Classes (cont.)}

    \begin{itemize}
        \item Instance definitions can itself be constrained and do recursively
            compose. Recall our ADT \texttt{List}:

           \begin{lstlisting}
    instance Ord a => Ord (List a) where
        Nil <= _    = True
        (List x xs) <= (List y ys) =
            if (x <= y) 
                then if (x == y) 
                    then xs <= ys
                    else True
                else False
           \end{lstlisting}

        \item<2-> And there is a similar instance for \texttt{[a]}
        \item<3-> That means that if we provide instance \texttt{Ord OurData} we get 
            \texttt{Ord [OurData]} for free.
    \end{itemize}
\end{frame}

\begin{frame}[fragile]
    \frametitle{Type Classes (cont.)}

    \begin{itemize}
        \item Some standard Haskell type classes:
            \begin{itemize}
                \item \texttt{Eq a} -- types with equality, \texttt{(==)}
                \item \texttt{Ord a} -- ordered types, \texttt{(<), (<=)}
                \item \texttt{Show a} -- types that can be pretty printed using
                    \texttt{show}
                \item \texttt{Num a} -- numeric types - \texttt{(+), (-), (*), abs,
                    signum}

                \item And many more \dots
            \end{itemize}

        \item<2-> Now we can fully understand type of e.g. \texttt{(+)}:

            \begin{verbatim}
GHCi, version 7.10.3: 
Prelude> :t (+)

(+) :: Num a => a -> a -> a
            \end{verbatim}
    \end{itemize}
\end{frame}

\begin{frame}[fragile]
    \frametitle{Strong Static Typing}

    \begin{itemize}
        \item So far, we always provided a type of top-level definitions
            (functions)

        \item \dots we did not provide a type of functions in e.g. where blocks

        \item<2-> Compiler \emph{infers} the \textbf{most generic} type of any 
            expression automatically and in fact we do not need to provide even
            types of top level definitions, e.g. for

            \begin{alltt}
    -- mySum :: ???
    mySum = foldr (+) 0 xs
            \end{alltt}
            \vskip -1em
            compiler infers the following type:

            \begin{alltt}
    GHCi, version 7.10.3:
    Prelude> :t mySum
    mySum :: (Num b, Foldable t) => t b -> b
            \end{alltt}
            \vskip -2em
        \item<3-> \textbf{Best Practice:} Do provide top-level types -- types
            document functions, and help compiler
            produce simpler error messages 

    \end{itemize}
\end{frame}

\begin{frame}
    \frametitle{Next time}

    \begin{itemize}
        \item Monday the the 22th of February, 2-3PM,
            Dalhousie 3G05 LT2
        
        \item More sorting algorithms
            \begin{itemize}
                \item Quick Sort
                \item Merge Sort
            \end{itemize}
        \item IO in Haskell (Monads)

    \end{itemize}

\end{frame}


\end{document}

\begin{frame}[fragile]
    \frametitle{Foo}

    \begin{itemize}
        \item
    \end{itemize}
\end{frame}

