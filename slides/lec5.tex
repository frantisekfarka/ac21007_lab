\documentclass[final,handout]{beamer}
%\documentclass[]{beamer}
\definecolor{green}{rgb}{0,0.6,0}
\definecolor{blue}{rgb}{0,0, 0.6}
\definecolor{red}{rgb}{0.6,0, 0}
\definecolor{gray}{rgb}{0.6,0.6, 0.6}
%\setbeamertemplate{headline}[verdeuni]
%\setbeamercovered{highly dinamic}
%\usepackage{eso-pic}

\usepackage[utf8]{inputenc}
\usepackage[english]{babel}
\usepackage{color}
\usepackage{url}
\usepackage{hyperref}
\usepackage{fancyvrb}
\usepackage{tikz}
\usepackage{alltt}
\usepackage{etex, xy}
\usepackage{tipa}

\xyoption{all} %para los diagramas
\usepackage{cancel}
\usepackage{verbatim}
%\usepackage{slashbox}
\usetikzlibrary{decorations.pathreplacing,shapes.arrows}
\newcommand\BackgroundPicture[1]{%
  \setbeamertemplate{background}{%
   \parbox[c][\paperheight]{\paperwidth}{%
      \vfill \hfill
\includegraphics[width=1\paperwidth,height=0.9\paperheight]{#1}
        \hfill \vfill
     }}}

\usepackage{color}
\usepackage{listings}


\newcommand{\rrdc}{\mbox{\,\(\Rightarrow\hspace{-9pt}\Rightarrow\)\,}} % Right reduction
\newcommand{\lrdc}{\mbox{\,\(\Leftarrow\hspace{-9pt}\Leftarrow\)\,}}% Left reduction
\newcommand{\lrrdc}{\mbox{\,\(\Leftarrow\hspace{-9pt}\Leftarrow\hspace{-5pt}\Rightarrow\hspace{-9pt}\Rightarrow\)\,}} % Equivalence
\DeclareMathOperator{\id}{Id}
\newcommand{\sCoq}{{\sc Coq}}
\newcommand{\SSReflect}{{\sc SSReflect}}
\newcommand{\Zset}{\mathbb{Z}}
\newcommand{\Bset}{\mathbb{Z}_2}


\newtheorem{Thm}{Theorem}
\newtheorem{Prop}{Proposition}
\newtheorem{Lem}{Lemma}
\newtheorem{Cor}{Corollary}
\newtheorem{Defn}{Definition}
\newtheorem{Goal}{Goal}
\newtheorem{GGoal}{General Goal}
\newtheorem{Alg}{Algorithm}
\newcommand{\inp}{{\itshape Input: }}
\newcommand{\outp}{{\itshape Output: }}

\renewcommand{\thefootnote}{$\ast$}
\title[Haskell Lecture 5]{AC21007: Haskell Lecture 5\\
    Selection Sort, Insertion Sort, and Bubble Sort
}
\author[Franti\v{s}ek Farka]{Franti\v{s}ek Farka}
\date{}
\usetikzlibrary{arrows}

\newcommand{\boxtt}[1]{\mbox{\scriptsize\texttt{#1}}}\newcommand{\cU}{\mathcal{U}}


\begin{document}
\BackgroundPicture{fondo1.png}
\begin{frame}
\titlepage
\end{frame}

\begin{frame}[fragile]
    \frametitle{Recapitulation}

    \begin{itemize}
        \item Function type \texttt{a -> b}
        \item Anonymous functions
        \item Currying
        \item Higher order functions
            \begin{itemize}
                \item {\tt map}, {\tt filter}
                \item Folds: {\tt foldr}, {\tt foldl}
            \end{itemize}
  \end{itemize}
\end{frame}  


\begin{frame}[fragile]
    \frametitle{Selection Sort}

    \begin{itemize}
        \item<1->[Goal:] We must devise an algorithm that sorts a collection of
            elements 

        \item<2->[Solution:] From those elements that are currently unsorted, find
            the smallest and place it next in the sorted collection.

        \item<3->[Example:] ~\\ {\tt
            ~~~~~~~~~~~~~~~~~~~~~\only<4-5>{$\color{red}\downarrow$}\\
            \only<3->{{[}]~~~~~~~~~~~ [7, 5, 2, 4]}~\\
            ~~~~~~~~~~~~~~~~~~~~~\only<7-8>{$\color{red}\downarrow$}\\
            \only<5->{{[}2]}~~~~~~~~~~~\only<6->{[7, 5, 4]}\\
            ~~~~~~~~~~~~~~~~~~\only<10-11>{$\color{red}\downarrow$}\\
            \only<8->{{[}2, 4]}~~~~~~~~\only<9->{[7, 5]}\\
            ~~~~~~~~~~~~~~~\only<13-14>{$\color{red}\downarrow$}\\
            \only<11->{{[}2, 4, 5]}~~~~~\only<12->{[7]}\\
            ~\\
            \only<14->{{[}2, 4, 5, 7]}~~\only<15->{[]}\\
        }
    \end{itemize}
\end{frame}


\begin{frame}[fragile]
    \frametitle{Selection Sort - C version}

    \begin{itemize}
        \item Implementation in C:

            \begin{verbatim}
void sel_sort(int* a, size_t n) {

    for (size_t i = 0, j; i < (n - 1); ++i) {
        j = i;

        for (size_t k = i + 1; k < n; ++k) {
            if (a[k] < a[j])
                j = k;

            /* int t = a[i]; a[i] =a[j]; a[j] = t; */
            swap(a, i, j);
        }
    }   
}
            \end{verbatim}
    \end{itemize}

\end{frame}

\begin{frame}[fragile]
    \frametitle{Selection Sort - Haskell version}

    \begin{itemize}
        \item[Goal:] \dots an algorithm that sorts a \emph{list} of
            elements 

        \item[Solution:] \dots from unsorted, find
            the smallest and place it next in the sorted list.
            \pause
            {\color{green}Empty list is trivially sorted!}

        \item<3->[Function:]
            \begin{alltt}
selSortImpl :: [Int] -> [Int] -> [Int]
selSortImpl sorted []  = sorted
selSortImpl sorted xs  = 
        selSortImpl (sorted ++ [x]) (removeFirst x xs)
     where
         x = minimum xs
         removeFirst _ [] = []
         removeFirst a (x:xs) = if x == a
            then xs
            else x : removeFirst a xs
    \end{alltt}
    \vskip -5em
    \only<4->{
    \begin{alltt}
selSort :: [Int] -> [Int]\\
selSort = selSortImpl []
    \end{alltt}
}
    \end{itemize}
\end{frame}

\begin{frame}[fragile]
    \frametitle{Insertion Sort}

    \begin{itemize}
        \item[Goal:] The same \dots

        \item[Solution:] From those elements that are currently unsorted, take 
            the \emph{first} and place it correctly in the sorted list.

        \item<2->[Example:] ~\\ {\tt
            ~\\
            \only<2->{{[}]}~~~~~~~~~~~~\only<2->{[7, 2, 5, 4]}\\
            ~\only<3->{$\color{red}\downarrow$}\\
            \only<3->{{[}7]}~~~~~~~~~~~\only<4->{[2, 5, 4]}\\
            ~\only<5->{$\color{red}\downarrow$}\\
            \only<5->{{[}2, 7]}~~~~~~~~\only<6->{[5, 4]}\\
            ~~~~\only<7->{$\color{red}\downarrow$}\\
            \only<7->{{[}2, 5, 7]}~~~~~\only<8->{[4]}\\
            ~~~~\only<9->{$\color{red}\downarrow$}\\
            \only<9->{{[}2, 4, 5, 7]}~~\only<10->{[]}\\
        }
    \end{itemize}
\end{frame}

\begin{frame}[fragile]
    \frametitle{Insertion Sort - Haskell version}

    \begin{itemize}
        \item[Function:]
            \begin{verbatim}
insSortImpl :: [Int] -> [Int] -> [Int]
insSortImpl sorted []  = sorted
insSortImpl sorted (x:xs)  = 
        insSortImpl (insert x sorted) xs
     where
         insert y [] = [y]
         insert y (z:zs) = if y <= z
            then y : (z : zs)
            else z : (insert y zs)

insSort :: [Int] -> [Int]
insSort = insSortImpl []
    \end{verbatim}
    \end{itemize}
\end{frame}

\begin{frame}[fragile]
    \frametitle{Syntactic intermezzo: \texttt{let \dots in} expression}

    \begin{itemize}
        \item We know \texttt{where} syntax
        \item The \texttt{let \dots in} epression
            \begin{alltt}
    {\bf let} <pat\(\sb{1}\)> = <expr\(\sb{1}\)>
        <pat\(\sb{n}\)> = <expr\(\sb{n}\)> {\bf in} <expr>
            \end{alltt}
            is a ``local'' version -- variables that are bound in patterns $pat_1$ to
            $pat_n$ after evaluating expressions $expr_1$ to $expr_n$ are in scope in
            $expr$ 
        \item \dots \textbf{and} in $expr_1$ to $expr_n$ -- bindings may by recursive!
        \item The expresion has a value of $expr$.
        \item E.g.:
            \begin{alltt}
    \textbackslash x -> let (y, z) = x in y + z
    let x = 1 : x in x
            \end{alltt}
    \end{itemize}
\end{frame}


\begin{frame}[fragile]
    \frametitle{Bubble Sort}

    \begin{itemize}
        \item[Goal:] The same \dots

        \item<2->[Intuition:] In each iteration \emph{bubble up} the greatest element.
            But which one is it?

        \item<3->[Solution:] 
            Start with the first element and bubble up as long as it is the
            greates so far, once we saw greater, continue with that one!
        
            \pause
            In each iteration, \emph{one} element is placed (the greates), after
            $n$ iterations - $n$ elements placed!

        \item<4->[Example:] ~\\ {\tt \small
                
                ~\only<5>{$\color{red}\downarrow$}
                ~~~~~~~~~~~~~~\only<6>{$\color{red}\downarrow$}
                ~~~~~~~~~~~~~~\only<7>{$\color{red}\downarrow$}
                ~~~~~~~~~~~~~~\only<8>{$\color{red}\downarrow$}~\\

                [\underline{5,2},7,4]
                \only<6->{$\rightarrow$ [2,\underline{5,7},4]} 
                \only<7->{$\rightarrow$ [2,5,\underline{7,4}]}
                \only<8->{$\rightarrow$ [2,5,4,7]}~\\

                ~\only<9>{$\color{red}\downarrow$}
                ~~~~~~~~~~~~~~\only<10>{$\color{red}\downarrow$}
                ~~~~~~~~~~~~~~\only<11>{$\color{red}\downarrow$}
                ~~~~~~~~~~~~~~\only<12>{$\color{red}\downarrow$}~\\

                \only<9->{[\underline{2,5},4,{\color{gray}7}]}
                \only<10->{$\rightarrow$ [2,\underline{5,4},{\color{gray}7}]}
                \only<11->{$\rightarrow$ [2,4,\underline{5,{\color{gray}7}}]}
                \only<12->{$\rightarrow$ [2,4,5,{\color{gray}7}]}~\\
                
                ~\only<13>{$\color{red}\downarrow$}
                ~~~~~~~~~~~~~~\only<14>{$\color{red}\downarrow$}
                ~~~~~~~~~~~~~~\only<15>{$\color{red}\downarrow$}
                ~~~~~~~~~~~~~~\only<16>{$\color{red}\downarrow$}~\\

                \only<13->{[\underline{2,4},{\color{gray}5,7}]}
                \only<14->{$\rightarrow$ [2,\underline{4,{\color{gray}5}},7]}
                \only<15->{$\rightarrow$ [2,4,\underline{{\color{gray}5,7}}]}
                \only<16->{$\rightarrow$ [2,4,{\color{gray}5,7}]}~\\
                
                ~\only<17>{$\color{red}\downarrow$}
                ~~~~~~~~~~~~~~\only<18>{$\color{red}\downarrow$}
                ~~~~~~~~~~~~~~\only<19>{$\color{red}\downarrow$}
                ~~~~~~~~~~~~~~\only<20>{$\color{red}\downarrow$}~\\

                \only<17->{[\underline{2,{\color{gray}4}}{\color{gray},5,7}]}
                \only<18->{$\rightarrow$ [2,{\color{gray}\underline{4,5},7}]}
                \only<19->{$\rightarrow$ [2,{\color{gray}4,\underline{5,7}}]}
                \only<20->{$\rightarrow$ [2,{\color{gray}4,5,7}]}~\\
        }
    \end{itemize}
\end{frame}

\begin{frame}[fragile]
    \frametitle{Bubble Sort - Haskell version}

    \begin{itemize}
        \item[Function:]
            \begin{verbatim}
bubbleSortImpl :: Int -> [Int] -> [Int]
bubbleSortImpl 0 xs  = xs 
bubbleSortImpl n xs  =
        bubbleSortImpl (n - 1) (bubble xs)
     where
         bubble [] = []
         bubble (x : []) = x : []
         bubble (x : y : ys) = if x <= y
                then x : (bubble (y : ys))
                else y : (bubble (x : ys))

bubbleSort :: [Int] -> [Int]
bubbleSort xs = let n = length xs 
    in bubbleSortImpl n xs
    \end{verbatim}
    \end{itemize}
\end{frame}


\begin{frame}[fragile]
    \frametitle{Time complexity}

    \begin{itemize}
        \item Not that easy as with Turing Machine, RAM, or C
        \item<2-> Abstract, non-mutable structures, no (out-of-box) direct indexing:
            \begin{itemize}
                \item In C, for \texttt{ar} array, and \texttt{n} index \\
                    \begin{verbatim}
    ar[n]
                    \end{verbatim}
                    is a ``primitive" action, $\mathcal{O}(1)$!
                \item<3-> In Haskell, for \texttt{lst} list, and \texttt{n} index\\
                    \begin{verbatim}
    lst !! n
                    \end{verbatim}
                    is a function call to
                    \begin{verbatim}
    (!!) :: Int -> [a] -> a
    (x:_)  !! 0  = x
    (_:xs) !! i  = xs !! (i - 1)
                    \end{verbatim}
                    in time $\mathcal{O}(n)$
            \end{itemize}
    \end{itemize}

\end{frame}

\begin{frame}[fragile]
    \frametitle{Time complexity}

    \begin{itemize}
        \item Not that easy as with Turing Machine, RAM, or C
        \item<2-> Lazy evaluation
            \begin{itemize}
                \item In C
                    \begin{verbatim}
    int dummy_minimum(int* ar, size_t n)
    {
        sel_sort(ar, n); // runs in O(n^2)
        return arr[0];   // runs in O(1)
    }
                    \end{verbatim}
                    runs in $\mathcal{O}(n^2)$
                \item<3-> In Haskell
                    \begin{verbatim}
    dummyMinimum :: [Int] -> Int
    dummyMinimum xs =
        head (           -- runs in O(1)
            selSort xs   -- only first selection 
                         --   evaluated - in O(n) !
        )
                    \end{verbatim}
                    runs in $\mathcal{O}(n)$
            \end{itemize}
    \end{itemize}
\end{frame}

\begin{frame}[fragile]
    \frametitle{Time complexity}

    \begin{itemize}
        \item Not that easy as with Turing Machine, RAM, or C
        \item Abstract, non-mutable structures, no (out-of-box) direct indexing
        \item Lazy evaluation
        \item Some algorithms are naturally imperative, other
            are functional!
    \end{itemize}
\end{frame}

\begin{frame}
    \frametitle{Next time}

    \begin{itemize}
        \item Monday the the 15th of February, 2-3PM,
            Dalhousie 3G05 LT2
        
        \item Defined data types
        \item Ad-hoc polymorphism: Typeclasses

    \end{itemize}

\end{frame}



\end{document}

