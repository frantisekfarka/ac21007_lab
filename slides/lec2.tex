\documentclass[final,handout]{beamer}
\definecolor{green}{rgb}{0,0.6,0}
\definecolor{blue}{rgb}{0,0, 0.6}
%\setbeamertemplate{headline}[verdeuni]
%\setbeamercovered{highly dinamic}
%\usepackage{eso-pic}

\usepackage[utf8]{inputenc}
\usepackage[english]{babel}
\usepackage{color}
\usepackage{url}
\usepackage{hyperref}
\usepackage{fancyvrb}
\usepackage{tikz}
\usepackage{alltt}
\usepackage{etex, xy}
 \usepackage{tipa}

\xyoption{all} %para los diagramas
\usepackage{cancel}
\usepackage{verbatim}
%\usepackage{slashbox}
\usetikzlibrary{decorations.pathreplacing,shapes.arrows}
\newcommand\BackgroundPicture[1]{%
  \setbeamertemplate{background}{%
   \parbox[c][\paperheight]{\paperwidth}{%
      \vfill \hfill
\includegraphics[width=1\paperwidth,height=0.9\paperheight]{#1}
        \hfill \vfill
     }}}

\usepackage{color}
\usepackage{listings}


\newcommand{\rrdc}{\mbox{\,\(\Rightarrow\hspace{-9pt}\Rightarrow\)\,}} % Right reduction
\newcommand{\lrdc}{\mbox{\,\(\Leftarrow\hspace{-9pt}\Leftarrow\)\,}}% Left reduction
\newcommand{\lrrdc}{\mbox{\,\(\Leftarrow\hspace{-9pt}\Leftarrow\hspace{-5pt}\Rightarrow\hspace{-9pt}\Rightarrow\)\,}} % Equivalence
\DeclareMathOperator{\id}{Id}
\newcommand{\sCoq}{{\sc Coq}}
\newcommand{\SSReflect}{{\sc SSReflect}}
\newcommand{\Zset}{\mathbb{Z}}
\newcommand{\Bset}{\mathbb{Z}_2}


\newtheorem{Thm}{Theorem}
\newtheorem{Prop}{Proposition}
\newtheorem{Lem}{Lemma}
\newtheorem{Cor}{Corollary}
\newtheorem{Defn}{Definition}
\newtheorem{Goal}{Goal}
\newtheorem{GGoal}{General Goal}
\newtheorem{Alg}{Algorithm}
\newcommand{\inp}{{\itshape Input: }}
\newcommand{\outp}{{\itshape Output: }}

\renewcommand{\thefootnote}{$\ast$}
\title[Haskell Lecture 2: List functions, function polymorphism]{AC21007: Haskell Lecture 2\\
List functions, function polymorphism, non-strict semantics}
\author[Franti\v{s}ek Farka]{Franti\v{s}ek Farka}
\date{}
\usetikzlibrary{arrows}

\newcommand{\boxtt}[1]{\mbox{\scriptsize\texttt{#1}}}\newcommand{\cU}{\mathcal{U}}


\begin{document}
\BackgroundPicture{fondo1.png}
\begin{frame}
\titlepage
\end{frame}

\begin{frame}
    \frametitle{Recapitulation}

    {\bf Haskell} 

    \begin{itemize}
    \item purely functional
    
    \item non-strict (also lazy) semantics

    \item (strong) static typing
    
    \end{itemize}

\end{frame}

\begin{frame}[fragile]
    \frametitle{Recapitulation (cont.)}

    \begin{itemize}
        \item 
            Data types (\texttt{Bool, Int, String}, \dots) and
            data values (\texttt{True, False,} \dots, -1, 0, 1, \dots, "Hello
            World!", \dots ) \\
            \uncover<2->{{\color{green}begin with an upper case letter}}%
        \item Function and variable identifiers (power, neg, b, n) \\
            \uncover<2->{{\color{green} begin with a lower case letter}}%
        \item<3-> Variables in Haskell cannot be updated
        \item<4-> Function definition:
            \begin{itemize}
                \item a set of equations, LHS is a pattern, RHS is an expression
                \item value matches only itself (True matches True)
                \item variable matches any value ... and binds the variable to
                    the matched value
            \end{itemize}
  \end{itemize}

\end{frame}  


\begin{frame}[fragile]
    \frametitle{Recapitulation (cont.)}

    \begin{itemize}
        \item An example: logic and
            \begin{alltt}
    myAnd :: Bool -> Bool -> Bool \pause
    myAnd True  True   = True
    myAnd True  False  = False
    myAnd False True   = False
    myAnd False False  = False
            \end{alltt}
\pause
        \item Recall:
            \begin{itemize}
                \item value matches only itself (True matches True)
                \item variable matches any value ... and binds the variable to
                    the matched value
            \end{itemize}
  \end{itemize}

\end{frame}  

\begin{frame}[fragile]
    \frametitle{Recapitulation (cont.)}

    \begin{itemize}
        \item An example: logic and
            \begin{alltt}
    myAnd :: Bool -> Bool -> Bool
    myAnd True  True   = True
    myAnd a     b      = False
            \end{alltt}

        \item Recall:
            \begin{itemize}
                \item value matches only itself (True matches True)
                \item variable matches any value ... and binds the variable to
                    the matched value
            \end{itemize}
  \end{itemize}

\end{frame}  

\begin{frame}[fragile]
    \frametitle{Recapitulation (cont.)}

    \begin{itemize}
        \item An example: logic and
            \begin{verbatim}
    myAnd :: Bool -> Bool -> Bool
    myAnd True  True   = True
    myAnd _     _      = False
            \end{verbatim}

        \item Recall:
            \begin{itemize}
                \item value matches only itself (True matches True)
                \item variable matches any value ... and binds the variable to
                    the matched value
            \end{itemize}
        \item New:
            \begin{itemize}
                \item {\color{green} '\texttt{\_}' matches any value, no binding
                    created}
            \end{itemize}
  \end{itemize}

\end{frame}  



\begin{frame}[fragile]
    \frametitle{List Datatype}

    \begin{itemize}
        \item<1-> data type \texttt{[Int]} -- a list where each element is of
            the type \texttt{Int}

        \item<2-> list values created by \emph{constructors}
            \begin{itemize}
                \item \texttt{[]} -- constructs an empty list, and
                \item \texttt{(:)} -- (\emph{cons}) from a value and list of values constructs
                    a new list, prepends the value
            \end{itemize}

        \item<3-> These are lists:
            \begin{alltt}
        []
        (1 : [])
        (2 : (5 : (3 : [])))
            \end{alltt}
        \item<4-> There is a special syntax:
            \begin{alltt}
        [1]
        [2, 5, 3]
            \end{alltt}

    \end{itemize}
\end{frame}

\begin{frame}[fragile]
    \frametitle{List Datatype (cont.)}

    \begin{itemize}
        \item<1-> data type \texttt{[Bool]} -- each element is of the type \texttt{Bool}
        \item<2-> yet again, constructors \texttt{[]} and \texttt{(:)}
        \item<3-> these are lists of booleans:
            \begin{alltt}
        []
        True : (False : (True : []))
        [False, True, True, False]
            \end{alltt}
    \end{itemize}
\end{frame}

\begin{frame}[fragile]
    \frametitle{Programming with list datatypes}

    \begin{itemize}
        \item<1-> The \texttt{sum} function computes the sum of a list of integers:
            \begin{alltt}
    sum :: [Int] -> Int
    sum []          = 0
    sum (x : xs)    = x + (sum xs)
            \end{alltt}

        \item<3-> The \texttt{all} function determines whether all the elements of a list
            of booleans are \texttt{True}:
            \begin{alltt}
    all :: [Bool] -> Bool
    all []          = True
    all (True : xs) = all xs
    all _           = False
            \end{alltt}

        \item<2-> {\color{green} New patterns:} list values can be matched against list
            constructors: \texttt{[]} matches itself and \texttt{(:)} matches
            a non-empty list, while matching both the patterns for the first element and
            for the rest of the list

    \end{itemize}
\end{frame}




\begin{frame}[fragile]
    \frametitle{Programming with list datatypes (cont.)}

    \begin{itemize}
        \item<1-> The \texttt{lengthInt} function computes the length of a list of integers:

    \begin{alltt}
    lengthInt :: [Int] -> Int
    lengthInt []       = 0
    lengthInt (_ : xs) = 1 + lengthInt xs
    \end{alltt}

        \item<2-> The \texttt{lengthBool} function computes the length of a list of integers:
    \begin{alltt}
    lengthBool :: [Bool] -> Int
    lengthBool []       = 0
    lengthBool (_ : xs) = 1 + lengthBool xs
    \end{alltt}

        \item<3-> The source code is nearly the same \dots \uncover<4->{can  we
            abstract over Int and Bool?}

    \end{itemize}

\end{frame}

\begin{frame}[fragile]
    \frametitle{List Datatype - \texttt{[a]}}

    \begin{itemize}
        \item<1-> Haskell has \emph{type variables} -- identifiers beginning
            with a lowercase letter
        \item<2-> Data type \texttt{[a]} -- a list where each element is of
            type \texttt{a}
        \item<3-> Exactly two constructors:
            \begin{itemize}
                \item \texttt{[] :: [a]}                   
                \item \texttt{(:) :: a -> [a] -> [a]}                   
            \end{itemize}
        \item<4-> A type with type variables is \emph{polymorphic}, it is
            instantiated to a \emph{monomorphic} type 
        \item<5-> A polymorphic \texttt{length} function:
            \begin{alltt}
    length :: [a] -> Int
    length []       = 0
    length (_ : xs) = 1 + length xs
            \end{alltt}
    \end{itemize}
\end{frame}

\begin{frame}[fragile]
    \frametitle{List Datatype \text{[a]} - some functions}

    \begin{itemize}
        \item<1-> \texttt{head} - access the first element:
    \begin{alltt}
    head :: [a] -> a 
    
    head (x : _) = x
    \end{alltt}

        \item<2-> \texttt{tail} - access the rest of a list:
    \begin{alltt}
    tail :: [a] -> [a]
    
    tail (_ : xs) = xs
    \end{alltt}

    \item<3-> What about a head of an empty list \texttt{head []}? \\
        ~\\
        \onslide<4->{\texttt{ \color{red}
            Error: Non-exhaustive patterns in function head}}

    \end{itemize}

    %picture - LYAH?

\end{frame}

\begin{frame}[fragile]
    \frametitle{List Datatype \text{[a]} - some functions}

    \begin{itemize}
        \item \texttt{head} - access the first element:
    \begin{alltt}
    head :: [a] -> a 
    head []      = ???
    head (x : _) = x
    \end{alltt}

        \item \texttt{tail} - access the rest of a list:
    \begin{alltt}
    tail :: [a] -> [a]
    tail []       = ???
    tail (_ : xs) = xs
    \end{alltt}

    \item What is the RHS? We don't know anything about the type \texttt{a}.

    \end{itemize}

    %picture - LYAH?

\end{frame}

\begin{frame}[fragile]
    \frametitle{List Datatype \text{[a]} - some functions}

    \begin{itemize}
        \item \texttt{head} - access the first element:
    \begin{alltt}
    head :: [a] -> a 
    head []      = error "Empty list"
    head (x : _) = x
    \end{alltt}

        \item \texttt{tail} - access the rest of a list:
    \begin{alltt}
    tail :: [a] -> [a]
    tail []       = error "Empty list" 
    tail (_ : xs) = xs
    \end{alltt}

    \item Haskell has special functions for run-time errors:

        \begin{itemize}
            \item \texttt{error :: String -> a} \\prints a specified error and
                terminates evaluation (program)
            \item \texttt{undefined :: a} \\ print a generic error and terminates
                evaluation
        \end{itemize}

    \end{itemize}

\end{frame}



\begin{frame}[fragile]
    \frametitle{Syntactic intermezzo -- functions and operators}

    \begin{itemize}
        \item Sometimes we do not want functions (e.g. \texttt{power},
            \texttt{sum}) but operators (e.g. \texttt{*}, \texttt{++})

        \item<2-> Consider the following list index function:

    \begin{alltt}
    at :: [a] -> Int -> a
    at (x : _)      0    = x
    at (_ : xs)     1    = at (i - 1) xs
    at []           _    = error "out of bound"

        -- usage:   at [1,2,3] 1     ==> 2
    \end{alltt}

    \vspace {-1em}
        \item<3-> We can use an operator:

    \begin{alltt}
    (!!) :: [a] -> Int -> a
    xs !! i = at xs i

        -- usage:   [1,2,3] !! 1    ==> 2
    \end{alltt}

    \end{itemize}

\end{frame}

\begin{frame}[fragile]
    \frametitle{Syntactic intermezzo -- functions and operators (cont.)}

    \begin{itemize}
        \item<1-> Function identifiers
            \begin{itemize}
                \item consist of a lowercase letter followed by zero or more
                    letters, digits, underscores, and single quotes
                \item prefix application (e.g. {\color{blue} \texttt{at [1,2,3]
                    0}})
            \end{itemize}
        \item<2-> Operators
            \begin{itemize}
                \item consist of symbols -- 
                    \texttt{\%!\#\$\%\&*+./<=>?\@\^{}|-\~{}}
                        
                    \item infix application (e.g. {\color{blue} \texttt{[1,2,3]
                        !! 0}})
            \end{itemize}
        \item<3-> Special syntax for using an operator in the prefix notation
            \begin{alltt}
        (!!) [1,2,3] 2
            \end{alltt}

        \item<4-> Special syntax for using a function in the infix notation
            \begin{alltt}
        [1,2,3] `at` 2
            \end{alltt}

    \end{itemize}

\end{frame}

\begin{frame}
    \frametitle{Next time}

    \begin{itemize}
        \item Monday the the 25th of January, 2-3PM,
            Dalhousie 3G05 LT2

        \item Non-strict semantics
        \item More list functions
        \item Tuples
        \item First-class functions
        \item Folds over lists
    \end{itemize}

\end{frame}




\end{document}

