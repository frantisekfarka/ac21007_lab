\documentclass[final,handout]{beamer}
\definecolor{green}{rgb}{0,0.6,0}
\definecolor{blue}{rgb}{0,0, 0.6}
%\setbeamertemplate{headline}[verdeuni]
%\setbeamercovered{highly dinamic}
%\usepackage{eso-pic}

\usepackage[utf8]{inputenc}
\usepackage[english]{babel}
\usepackage{color}
\usepackage{url}
\usepackage{hyperref}
\usepackage{fancyvrb}
\usepackage{tikz}
\usepackage{alltt}
\usepackage{etex, xy}
 \usepackage{tipa}

\xyoption{all} %para los diagramas
\usepackage{cancel}
\usepackage{verbatim}
%\usepackage{slashbox}
\usetikzlibrary{decorations.pathreplacing,shapes.arrows}
\newcommand\BackgroundPicture[1]{%
  \setbeamertemplate{background}{%
   \parbox[c][\paperheight]{\paperwidth}{%
      \vfill \hfill
\includegraphics[width=1\paperwidth,height=0.9\paperheight]{#1}
        \hfill \vfill
     }}}

\usepackage{color}
\usepackage{listings}


\newcommand{\rrdc}{\mbox{\,\(\Rightarrow\hspace{-9pt}\Rightarrow\)\,}} % Right reduction
\newcommand{\lrdc}{\mbox{\,\(\Leftarrow\hspace{-9pt}\Leftarrow\)\,}}% Left reduction
\newcommand{\lrrdc}{\mbox{\,\(\Leftarrow\hspace{-9pt}\Leftarrow\hspace{-5pt}\Rightarrow\hspace{-9pt}\Rightarrow\)\,}} % Equivalence
\DeclareMathOperator{\id}{Id}
\newcommand{\sCoq}{{\sc Coq}}
\newcommand{\SSReflect}{{\sc SSReflect}}
\newcommand{\Zset}{\mathbb{Z}}
\newcommand{\Bset}{\mathbb{Z}_2}


\newtheorem{Thm}{Theorem}
\newtheorem{Prop}{Proposition}
\newtheorem{Lem}{Lemma}
\newtheorem{Cor}{Corollary}
\newtheorem{Defn}{Definition}
\newtheorem{Goal}{Goal}
\newtheorem{GGoal}{General Goal}
\newtheorem{Alg}{Algorithm}
\newcommand{\inp}{{\itshape Input: }}
\newcommand{\outp}{{\itshape Output: }}

\renewcommand{\thefootnote}{$\ast$}
\title[Haskell Lecture 3: TBD]{AC21007: Haskell Lecture 3\\
    Non-strict semantics, tuples
}
\author[Franti\v{s}ek Farka]{Franti\v{s}ek Farka}
\date{}
\usetikzlibrary{arrows}

\newcommand{\boxtt}[1]{\mbox{\scriptsize\texttt{#1}}}\newcommand{\cU}{\mathcal{U}}


\begin{document}
\BackgroundPicture{fondo1.png}
\begin{frame}
\titlepage
\end{frame}

\begin{frame}[fragile]
    \frametitle{Recapitulation}

    \begin{itemize}
        \item Data type List (\texttt{[]}, \texttt{(:)})
        \item<2-> Function definition:
            \begin{itemize}
                \item<3-> a set of equations:\\
                    \texttt{<identifier> <pat$_1$> \dots ~<pat$_n$> = <expr>} 
                \item<4-> patterns:
                    \begin{itemize}
                        \item a value (\texttt{True, False, 0, \dots})
                        \item a variable (\texttt{x, xs, myVariable, \dots})
                        \item \texttt{\_} -- wildcard, "don't care" pattern
                        \item list constructors, i.e.: 
                            \texttt{[], (<pat$_{head}$> : <pat$_{tail}$ )}
                    \end{itemize}
            \end{itemize}
  \end{itemize}

  \onslide<5->{Demo \dots}

\end{frame}  



%% lazy evaluation -- division by zero
\begin{frame}[fragile]
    \frametitle{Non-strict (lazy) semantics}

    \begin{itemize}
        \item<1-> In Haskell, expressions are evaluated lazily -- not evaluated until
            needed

        \item<2-> Consider a variant of our \texttt{power} function:
            \begin{alltt}
    power' :: Int -> Int -> Float -> Int
    power' b 0 _ = 1
    power' b n x = b * (power b (n - 1) x)
            \end{alltt}


    \item<3-> Consider the following function call: \\
        \vskip 1em
\pause
\pause
    {\tt \small
        power' 7 2 (1.0 / 0) \\ 
        \pause $==>$ 7 * (power' 7 (2 - 1) (1.0 / 0)) \\
        \pause $==>$ 7 * (power' 7 1) (1.0 / 0) \\ 
        \pause $==>$ 7 * (7 * (power' 7 (1 - 1) (1.0 / 0))) \\
        \pause $==>$ 7 * (7 * (power' 7 0 (1.0 / 0))) \\
        \pause $==>$ 7 * (7 * (1)) \\
        \dots \\
        $==>$ 49
    }


    \end{itemize}

\end{frame}

%% lazy evaluation -- division by zero
\begin{frame}[fragile]
    \frametitle{Non-strict (lazy) semantics - infinite lists}

    \begin{itemize}
        \item<1-> Consider the following function:
            \begin{alltt}
    repeat :: a -> [a]
    repeat x    =   x : (repeat x)
            \end{alltt}
        \pause
            this function defines an infinite list of elements, e.g:
            \begin{alltt}
    repeat 1    ==> [1, 1, 1, 1, 1, 1, ... ]
            \end{alltt}
    \end{itemize}
\end{frame}

\begin{frame}[fragile]
    \frametitle{Non-strict (lazy) semantics - infinite lists (cont.)}

    \begin{itemize}
        \item<1-> A more useful example -- powers of an integer:
            \begin{alltt}
    powersof :: Integer -> [Integer]
    powersof b  = pow b 1
        where
            pow b p = p : pow b (b * p)
            \end{alltt}
            \onslide<3->{
            this function defines an infinite list, e.g.:
            \begin{alltt}
    powersof 2    ==> [1, 2, 4, 8, 16, 32, ... ]
            \end{alltt}
        }
        \item<4-> Our \texttt{power} function:
            \begin{alltt}
    power :: Integer -> Integer -> Integer
    power b n = (powersof b) !! n
            \end{alltt}

        
        \item<2-> {\color{green} Note:}
            \begin{itemize}
                \item \texttt{Int} is machine integer (32/64 bits),
                    \texttt{Integer} is arbitrary precision integer
                \item \texttt{where} block allows for local-scope definitions

            \end{itemize}
    \end{itemize}

\end{frame}

\begin{frame}[fragile]
    \frametitle{Tuple Datatype -- \texttt{(a, b)}}

    \begin{itemize}
        \item<1-> Data type \texttt{(a, b)} -- type of pairs of values, polymorphic
            in both of its components \texttt{a} and \texttt{b}
        \item<2-> One constructor \texttt{(a, b) :: a -> b -> (a, b)}
        \item<3-> E.g. \texttt{(True, "hello") :: (Bool, String)}
        \item<4-> Functions (projections) \texttt{fst} and \texttt{snd}:

            \begin{alltt}
        fst :: (a, b) -> a
        fst (x, _) = x

        snd :: (a, b) -> b
        snd (_, y) = y
            \end{alltt}

        \item<5-> {\color{green} Note:} tuple constructor may be used as a pattern
        \item<6-> There are also triples \texttt{(a, b, c)}, quadruples \texttt{(a,
            b, c, d)}, etc. (no genetic \texttt{fst} and \texttt{snd} though)
    \end{itemize}


\end{frame}


\begin{frame}[fragile]
    \frametitle{Combining lists and tuples -- \texttt{zip}}

    \begin{itemize}
        \item \texttt{zip} takes two lists and returns a list of corresponding pairs
        \item If one input list is short, excess elements of the longer list are
            discarded

    \pause
    \end{itemize}
            \begin{alltt}
    zip :: [a] -> [b] -> [(a,b)]
    zip []     _      = []
    zip _      []     = []
    zip (a:as) (b:bs) = (a,b) : zip as bs
        \end{alltt}
\end{frame}

\begin{frame}[fragile]
    \frametitle{Syntactic intermezzo: \texttt{if then else}}

    \begin{itemize}
        \item Haskell has a conditional \textbf{expression}:\\
            ~\\
            \texttt{{\bf if} <cnd \onslide<2->{:: Bool}> 
                {\bf then} <x \onslide<3->{:: a}> 
                {\bf else} <y \onslide<3->{:: a}>
                ~~~~~~~~~~~~\onslide<4->{:: a}}
        \item<2-> \texttt{<cnd>} is an expression that evaluates to \texttt{Bool}
        \item<3-> Both branches are expressions that evaluates to a value 
            of a type \texttt{a}
        \item<4-> The whole expression evaluates to the appropriate value
            of a type \texttt{a}
        \item<5-> \texttt{then} and \texttt{else} branches may be indented by
            white-space
    \end{itemize}
\end{frame}

\begin{frame}[fragile]
    \frametitle{Syntactic intermezzo: \texttt{if then else} (cont.)}

    \begin{itemize}
        \item \texttt{if} is an expression -- it can be used as such, e. g.:

    \begin{alltt}
ghci> (if ("a" == "b") then 3 else 5) + 2
    ==> 7
    \end{alltt}

\item<2->In function definition (note the indentation):

    \begin{alltt}
    max :: Int -> Int -> Int
    max x y = if x > y  then x
                        else y
    \end{alltt}
    \end{itemize}
\end{frame}



%\begin{frame}[fragile]
%    \frametitle{Anonymous (lambda) functions}
%
%    \begin{itemize}
%        \item<3-> Functions without a name
%        \item<3->  Syntax:
%            \begin{alltt}
%    \textbackslash{}<var\(\sb{1}\)> \dots <var\(\sb{n}\)> -> <expr>
%            \end{alltt}
%        \item<4-> Variables $var_1$ to $var_n$ in scope in the expression $expr$
%            
%        \item<5-> Anonymous functions:
            
%            \begin{itemize}
%                \item<5-> can be applied to an argument:\\
%                    \texttt{(\textbackslash x -> 2 + x) 3 ==> 5}
%                \item<6-> can be passed as an argument \dots functions \textbf{are}
%                    values
%            \end{itemize}
%
%        \item<1-> \onslide<7->{E.g.:}\\
%    \texttt{                ~~~~~2 + 3 :: Int}\\
%    \texttt{\onslide<3->{\textbackslash{}x ->} 2 + x :: \onslide<3->{Int ->} Int}\\
%    \texttt{\onslide<2>{~~~Not in scope: `x`}}
%
%    \end{itemize}
%
%\end{frame}

%\begin{frame}[fragile]
%    \frametitle{Anonymous (lambda) functions (cont.)}
%
%    \begin{itemize}
%        \item \texttt{filter}, applied to a predicate and a list, returns the list of
%            those elements that satisfy the predicate
%    \pause
%        \begin{alltt}
%    filter :: (a -> Bool) -> [a] -> [a]
%    filter _    []     = []
%    filter pred (x:xs) = if (pred x)
%        then x : filter pred xs
%        else filter pred xs
%        \end{alltt}
%
%        \item<3-> E.g:
%        \begin{verbatim}
%    filter (\x -> x `mod` 2 == 1) [1, 2, 3, 4, 5, 6]
%        ==> [1, 3, 5, 7]
%
%    filter (\x -> x `mod` 2 == 0) [1, 2, 3, 4, 5, 6]
%        ==> [2, 4, 6]
%        \end{verbatim}
%    \end{itemize}
%
%\end{frame}

%\begin{frame}[fragile]
%    \frametitle{First-class functions}
%
%    \begin{itemize}
%        \item All functions can be passed as arguments, 
%            e.g standard functions \texttt{even} and \texttt{odd}:
%        \begin{verbatim}
%    filter even [1, 2, 3, 4, 5, 6]
%        ==> [1, 3, 5, 7]
%
%    filter even [1, 2, 3, 4, 5, 6]
%        ==> [2, 4, 6]
%        \end{verbatim}
%%        \item Function type \texttt{a -> b} (right-associative)
%%        \item Values of this types constructed by
%%            \begin{itemize}
%%                \item usual function definition
%%                \item lambda construction
%%            \end{itemize}
%
%
%
%    \end{itemize}
%
%\end{frame}

%\begin{frame}[fragile]
%    \frametitle{First-class functions (cont)}
%
%    \begin{itemize}
%        \item<1-> Function type \texttt{a -> b} (right-associative)
%        \item<2-> Values of this type are constructed by
%            \begin{itemize}
%                \item<3-> usual function definitions
%                \item<3-> lambda constructions
%            \end{itemize}
%
%
%        \item<4-> \onslide<9->{The following definitions of \texttt{max} are
%            equivalent:}\\
%            ~\\
%            \texttt{~~~max :: \onslide<7->{(}Int -> \onslide<5->{(}Int ->
%            Int\onslide<5->{)}\onslide<7->{)}}\\
%            \texttt{\onslide<5->{--} max x y = if x > y then x else y}\\
%            \texttt{\onslide<7->{--} \onslide<5->{max x =}
%                \onslide<6->{\textbackslash y ->
%            if x > y then x else y}}\\
%            \texttt{~~~\onslide<7->{max =}
%                \onslide<8->{\textbackslash x y ->
%            if x > y then x else y}}\\
%
%        \vskip 1em
%
%        \item<10-> Haskell compiler will figure out types from
%            LHS patterns and type of RHS expression
%        \item<11-> {\color{green}Note:} In a function definition all equations
%            must have the same number of LHS patterns
%
%
%    \end{itemize}
%
%\end{frame}





\begin{frame}
    \frametitle{Next time}

    \begin{itemize}
        \item Monday the the 1st of February, 2-3PM,
            Dalhousie 3G05 LT2
        
        \item Anonymous functions
        \item Higher order functions
        \item More (higher-order) list functions (\texttt{map}, \dots)
        \item Recursion, folds over lists

    \end{itemize}

\end{frame}




\end{document}

