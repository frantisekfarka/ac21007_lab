\documentclass[final,handout]{beamer}
\definecolor{green}{rgb}{0,0.6,0}
\definecolor{blue}{rgb}{0,0, 0.6}
%\setbeamertemplate{headline}[verdeuni]
%\setbeamercovered{highly dinamic}
%\usepackage{eso-pic}

\usepackage[utf8]{inputenc}
\usepackage[english]{babel}
\usepackage{color}
\usepackage{url}
\usepackage{hyperref}
\usepackage{fancyvrb}
\usepackage{tikz}
\usepackage{alltt}
\usepackage{etex, xy}
 \usepackage{tipa}

\xyoption{all} %para los diagramas
\usepackage{cancel}
\usepackage{verbatim}
%\usepackage{slashbox}
\usetikzlibrary{decorations.pathreplacing,shapes.arrows}
\newcommand\BackgroundPicture[1]{%
  \setbeamertemplate{background}{%
   \parbox[c][\paperheight]{\paperwidth}{%
      \vfill \hfill
\includegraphics[width=1\paperwidth,height=0.9\paperheight]{#1}
        \hfill \vfill
     }}}

\usepackage{color}
\usepackage{listings}


\newcommand{\rrdc}{\mbox{\,\(\Rightarrow\hspace{-9pt}\Rightarrow\)\,}} % Right reduction
\newcommand{\lrdc}{\mbox{\,\(\Leftarrow\hspace{-9pt}\Leftarrow\)\,}}% Left reduction
\newcommand{\lrrdc}{\mbox{\,\(\Leftarrow\hspace{-9pt}\Leftarrow\hspace{-5pt}\Rightarrow\hspace{-9pt}\Rightarrow\)\,}} % Equivalence
\DeclareMathOperator{\id}{Id}
\newcommand{\sCoq}{{\sc Coq}}
\newcommand{\SSReflect}{{\sc SSReflect}}
\newcommand{\Zset}{\mathbb{Z}}
\newcommand{\Bset}{\mathbb{Z}_2}


\newtheorem{Thm}{Theorem}
\newtheorem{Prop}{Proposition}
\newtheorem{Lem}{Lemma}
\newtheorem{Cor}{Corollary}
\newtheorem{Defn}{Definition}
\newtheorem{Goal}{Goal}
\newtheorem{GGoal}{General Goal}
\newtheorem{Alg}{Algorithm}
\newcommand{\inp}{{\itshape Input: }}
\newcommand{\outp}{{\itshape Output: }}

\renewcommand{\thefootnote}{$\ast$}
\title[Haskell Lecture 3: TBD]{AC21007: Haskell Lecture 4\\
    Higher order functions, map, folds
}
\author[Franti\v{s}ek Farka]{Franti\v{s}ek Farka}
\date{}
\usetikzlibrary{arrows}

\newcommand{\boxtt}[1]{\mbox{\scriptsize\texttt{#1}}}\newcommand{\cU}{\mathcal{U}}


\begin{document}
\BackgroundPicture{fondo1.png}
\begin{frame}
\titlepage
\end{frame}

\begin{frame}[fragile]
    \frametitle{Recapitulation}

    \begin{itemize}
        \item Data type tuple \texttt{(a, b)}
        \item Non-strict semantics:
            \begin{itemize}
                    \item expressions evaluated on-demand
                    \item allows infinite data structures (lists)
            \end{itemize}
  \end{itemize}
\end{frame}  





\begin{frame}[fragile]
    \frametitle{Anonymous (lambda) functions}

    \begin{itemize}
        \item<4-> Functions without a name
        \item<4->  Syntax:
            \begin{alltt}
    \textbackslash{}<var\(\sb{1}\)> \dots <var\(\sb{n}\)> -> <expr>
            \end{alltt}
        Variables $var_1$ to $var_n$ in scope in the expression $expr$
            
        \item<5-> Anonymous functions:
            
            \begin{itemize}
                \item<6-> can be applied to an argument:\\
                    \texttt{(\textbackslash x -> 2 + x) 3 ==> 5}
                \item<7-> can be passed as an argument \\
                        \dots anonymous functions \textbf{are} values
            \end{itemize}

        \item<2-> \onslide<8->{E.g.:}\\
    \texttt{                ~~~~~2 + 3 :: Int}\\
    \texttt{\onslide<4->{\textbackslash{}x ->} 2 + x :: \onslide<4->{Int ->} Int}\\
    \texttt{\onslide<3>{~~~Not in scope: `x`}}

    \end{itemize}

\end{frame}

\begin{frame}[fragile]
    \frametitle{Anonymous (lambda) functions (cont.)}

    \begin{itemize}
        \item \texttt{filter}, applied to a predicate and a list, returns the list of
            those elements that satisfy the predicate
    \pause
        \begin{alltt}
    filter :: (a -> Bool) -> [a] -> [a]
    filter _    []     = []
    filter pred (x:xs) = if (pred x)
        then x : filter pred xs
        else filter pred xs
        \end{alltt}

        \item<3-> E.g:
        \begin{verbatim}
    filter (\x -> x `mod` 2 == 1) [1, 2, 3, 4, 5, 6]
        ==> [1, 3, 5]

    filter (\x -> x `mod` 2 == 0) [1, 2, 3, 4, 5, 6]
        ==> [2, 4, 6]
        \end{verbatim}
    \end{itemize}

\end{frame}

\begin{frame}[fragile]
    \frametitle{First-class functions}

    \begin{itemize}
        \item All functions can be passed as an argument, 
            e.g standard functions \texttt{even} and \texttt{odd}:
        \begin{verbatim}
    filter odd [1, 2, 3, 4, 5, 6]
        ==> [1, 3, 5]

    filter even [1, 2, 3, 4, 5, 6]
        ==> [2, 4, 6]
        \end{verbatim}
        \item<2-> All functions are just values
        \item<3-> We will call functions that take a function as an argument
            \emph{higher order functions}
%        \item Function type \texttt{a -> b} (right-associative)
%        \item Values of this types constructed by
%            \begin{itemize}
%                \item usual function definition
%                \item lambda construction
%            \end{itemize}



    \end{itemize}

\end{frame}

\begin{frame}[fragile]
    \frametitle{Some useful higher order functions}
    \begin{itemize}
        \item<1-> \texttt{map} - applies a function to each element of a list
            \begin{alltt}
    map :: (a -> b) -> [a] -> [b]
    map _ []     = []
    map f (x:xs) = f x : map f xs
\pause
map (\textbackslash{}x -> 2 * x)) [1, 2, 3, 4]
    ==> [2, 4, 6, 8]
            \end{alltt}

        \item<3-> \texttt{zipWith} - generalises \texttt{zip}, combines list elements
            with the function in its first argument, truncates the longer list
            \begin{alltt}
    zipWith :: (a -> b -> c) -> [a] -> [b] -> [c]
    zipWith _ []     _      = []
    zipWith _ _    []       = []
    zipWith f (a:as) (b:bs) = f a b : zipWith f as bs
\pause
zipWith (+) [2, 3, 4] [5, 6, 7]
    [7, 9, 11]
            \end{alltt}
    \end{itemize}
\end{frame}


\begin{frame}[fragile]
    \frametitle{First-class functions (cont)}

    \begin{itemize}
        \item<1-> Function type \texttt{a -> b} (right-associative)
        \item<2-> Values of this type are constructed by:
            \begin{itemize}
                \item<3-> the usual function definitions
                \item<3-> lambda constructions
            \end{itemize}


        \item<4-> \onslide<9->{The following definitions of \texttt{max} are
            equivalent:}\\
            ~\\
            \texttt{~~~max :: \onslide<7->{(}Int -> \onslide<5->{(}Int ->
            Int\onslide<5->{)}\onslide<7->{)}}\\
            \texttt{\onslide<5->{--} max x y = if x > y then x else y}\\
            \texttt{\onslide<7->{--} \onslide<5->{max x =}
                \onslide<6->{\textbackslash y ->
            if x > y then x else y}}\\
            \texttt{~~~\onslide<7->{max =}
                \onslide<8->{\textbackslash x y ->
            if x > y then x else y}}\\

        \vskip 1em

        \item<10-> Haskell compiler will figure out types from
            LHS patterns and type of RHS expression
        \item<11-> {\color{green}Note:} In a function definition all equations
            must have the same number of LHS patterns


    \end{itemize}

\end{frame}

\begin{frame}[fragile]
    \frametitle{Currying}

    \begin{itemize}
        \item<1-> \textbf{currying} - translating the evaluation of a function that
            takes multiple arguments \onslide<3->{(a tuple of arguments)} into
            evaluating a sequence of (higher-order) functions, each with a single 
            argument
        \item<2-> A variant of \texttt{max}:
            \begin{alltt}
    max' :: (d, d) -> d
    max' (x, y) = if x > y then ...
            \end{alltt}

        \item<4-> We can express this translation as higher-order function:
            \begin{alltt}
    curry :: ((a, b) -> c) -> a -> b -> c
    curry f x y = f (x, y)
            \end{alltt}
            
        \item<5->There is also the reverse translation:
            \begin{alltt}
    uncurry :: (a -> b -> c) -> (a, b) -> c
    uncurry f (x, y) = f x y
            \end{alltt}
    \end{itemize}
\end{frame}




\begin{frame}[fragile]
    \frametitle{Function manipulation}

    \begin{itemize}
        \item<1-> Composition
            \begin{itemize}
                \item The usual $(f.g)(x) = f(g(x))$
                \item<2-> Operator \texttt{(.)}, higher order function:
                    \begin{alltt}
    (.) :: (b -> c) -> (a -> b) -> a -> c
    f . g = \textbackslash x -> f (g x)
                    \end{alltt}
                \item<3-> E.g.:
                    \begin{alltt}
    filter even . (filter (\textbackslash x -> x `mod` 3 == 0))  
                    \end{alltt}
            \end{itemize}
        \item<4-> Partial application
            \begin{itemize}
                \item We can provide function only with first n arguments
                \item Result is a partially applied function - a new function
                    taking the rest of arguments
                \item<5-> E.g: \texttt{max 5, (1 +), (2 *)}
            \end{itemize}
    \end{itemize}
\end{frame}

\begin{frame}[fragile]
    \frametitle{List folding}

    \begin{itemize}
        \item Let's compare two recursive functions on lists:

        \begin{itemize}
            \item Function \texttt{sum}:
            \begin{alltt}
    sum :: [Integer] -> Integer
    sum []           = 0
    sum (x : xs)     = x + sum xs
            \end{alltt}

        \item Function \texttt{maximum}:
            \begin{alltt}
    maximum :: [Integer] -> Integer
    maximum []       = error "empty list"
    maximum (x : []) = x
    maximum (x : xs) = max x (maximum xs)
            \end{alltt}
        \end{itemize}

    \item<2-> {\color{red}Recursive case has the same structure:}
            $$
            recf \quad (x:xs) \quad = \quad f \quad x \quad (recf \quad xs)
            $$
        %\item Base case is different \dots

    \end{itemize}
    ~
    \\~
\end{frame}

\begin{frame}[fragile]
    \frametitle{List folding}

    \begin{itemize}
        \item Let's compare two recursive functions on lists:

        \begin{itemize}
            \item Function \texttt{sum}:
            \begin{alltt}
    sum :: [Integer] -> Integer
    sum []           = 0
    {\color{green}sum (x : xs)     = (+) x (sum xs)}
            \end{alltt}

        \item Function \texttt{maximum}:
            \begin{alltt}
    maximum :: [Integer] -> Integer
    maximum []       = error "empty list"
    maximum (x : []) = x
    maximum (x : xs) = max x (maximum xs)
            \end{alltt}
        \end{itemize}

        \item Recursive case has the same structure:
            $$
            recf \quad (x:xs) \quad = \quad f \quad x \quad (recf \quad xs)
            $$
        \item<2-> Base case is different \dots

    \end{itemize}
\end{frame}


\begin{frame}[fragile]
    \frametitle{List folding (cont.)}

    \begin{itemize}
        \item Let's slightly modify our two functions:

            \item Function \texttt{sum}: \\
    {\tt\small
sum :: \onslide<10->{(Int -> Int -> Int) ->}\\
~~~~ \onslide<2->{Int -> }[Int] -> Int\\
sum~~~~ \onslide<11->{\_} \onslide<2->{val} []~~~~~~~ = \onslide<1>{0}\onslide<2->{val}\\
sum~~~~ \onslide<11->{f} \onslide<3->{val} (x : xs) =
\onslide<12->{f}\onslide<1-11>{(+)} x (sum \onslide<12->{f} \onslide<3->{val} xs)
\\~\\
\quad sum \onslide<13->{(+)} \onslide<4->{0} [1, 2, 3, 4, 5]
}
\\~
        \item Function \texttt{maximum}: \\
    {\tt\small
maximum :: \onslide<14->{(Int -> Int -> Int) ->}\\
~~~~ \onslide<5->{Int ->} [Int] -> Int\\
maximum \onslide<14->{\_} \onslide<6->{val} []~~~~~~~ =
\onslide<6->{val}\only<beamer>{\onslide<1-5>{error "..."}}\\
\only<beamer>{\onslide<1-6>{maximum~~~~~~ (x : []) = x}\\}
maximum \onslide<14->{f} \onslide<8->{val} (x : xs) =
\onslide<14->{f}\onslide<1-13>{max} x (maximum \onslide<14->{f} \onslide<8->{val} xs)\\
~\\
\quad maximum \onslide<14->{max} \onslide<9->{3}
[\only<beamer>{\onslide<1-8>{3,}} 2, 5, 4, 2]
}

%        \item Recursive case has the same structure:
%            $$
%            recf \quad (x:xs) \quad = \quad f \quad x \quad (recf \quad xs)
%            $$
%        \item Base case is different \dots

    \end{itemize}
\end{frame}

\begin{frame}[fragile]
    \frametitle{List folding - foldr and foldl}

    \begin{itemize}
        \item One generic function \texttt{foldr} for right-associative recursion:

            \begin{alltt}
    foldr :: (a -> b -> b) -> b -> [a] -> b
    foldr _ z []       = z
    foldr f z (x : xs) = f x (foldr f z)
            \end{alltt}

        \item<2-> The structure of recursion is\\
\quad \texttt{foldr f z [x$_1$, x$_2$, \dots, x$_n$]\\ 
    \quad~\quad==> f x$_1$ (f x$_2$ \dots (f x$_n$)\dots)
}

        \item<3-> There is also function\\

            \quad\texttt{foldl :: (b -> a -> b) -> b -> [a] -> b}\\

            for left-associative recursion, i.e.:\\~\\
\quad \texttt{foldl f z [x$_1$, x$_2$, \dots, x$_n$]\\ 
    \quad~\quad==> f x$_n$ (\dots(f x$_2$ (f x$_1$)\dots)
}
    \end{itemize}
\end{frame}

\begin{frame}[fragile]
    \frametitle{List folding - examples}

    \begin{itemize}
        \item Our \texttt{sum} and \texttt{maximum} as folds:

            \begin{alltt}
    sum :: [Int] -> Int
    sum xs = foldr (+) 0 xs

    maximum :: [Int] -> Int
    maximum []     = error "empty list"
    maximum (x:xs) = foldr max x xs
            \end{alltt}

        \item A fold where \texttt{a} and \texttt{b} are different:

            \begin{alltt}
    length :: [a] -> Integer
    length xs = foldr f 0 xs
        where
            -- f :: a -> Integer -> Integer
            f _ b = 1 + b
            \end{alltt}
    \end{itemize}
\end{frame}

%\begin{frame}[fragile]
%    \frametitle{Syntactic intermezzo: \texttt{if then else}}

%    \begin{itemize}
%        \item Haskell has a conditional \textbf{expression}:\\
%            ~\\
%            \texttt{{\bf if} <cnd \onslide<2->{:: Bool}> 
%                {\bf then} <x \onslide<3->{:: a}> 
%                {\bf else} <y \onslide<3->{:: a}>
%                ~~~~~~~~~~~~\onslide<4->{:: a}}
%    \end{itemize}
%\end{frame}



\begin{frame}
    \frametitle{Next time}

    \begin{itemize}
        \item Monday the the 8th of February, 2-3PM,
            Dalhousie 3G05 LT2
        
        \item Sorting algorithms on lists
            \begin{itemize}
                \item Selection Sort
                \item Insertion Sort
                \item Bubble Sort
            \end{itemize}
        %\item Anonymous functions
        %\item Higher order functions
        %\item More (higher-order) list functions (\texttt{map}, \dots)
        %\item Recursion, folds over lists

    \end{itemize}

\end{frame}




\end{document}

